%TODO
% Fotos Team
% Überarbeiten


\section{Projektmanagement und Organisation}
\label{sec:ProjUOrg}

\subsection{Team} %Fotos fehlen noch
\label{sec:Team}
	\subsubsection*{Dario Wagner}
		Verantwortlich für: 
		\begin{itemize}
			\item Parser
			\item iCal
		\end{itemize}
	\subsubsection*{Marcel Stering}
		Verantwortlich für: 
		\begin{itemize}
			\item Security
			\item Webseite
		\end{itemize}
	\subsubsection*{Matthias Franz}
		Verwantwortlich für: 
		\begin{itemize}
			\item iCal
			\item Datenbank
			\item Projektleitung
		\end{itemize}				

\subsection{Auftraggeber - Intact Systems}
\label{sec:Auftraggeber}
Unsere Diplomarbeit wurde im Auftrag des Unternehmens Intact Systems durchgeführt. Intact Systems ist eine in Lebring sitzende Softwareentwicklungsfirma welche sich auf Audits, Zertifizierungsmanagement, Rückverfolgbarkeit und Qualitätsmanagement spezialisiert hat auch Sitze in der USA und in der Schweiz. Unsere Ansprechpartner waren Rudolf Rauch und Mathias Schober. Intact bietet maßgeschneiderte Softwarelösungen und standardisierte. Intacts bekanntestes Produkt ist Ecert, welches interne Audits, Zertifizierung, Gütesiegel, Lieferanten und noch vieles mehr managen kann.
	\subsubsection*{Kontaktaufnahme mit Intact Systems}
	Mit Intact Systems wurde am Recruiting-Day der HTBLA Kaindorf Kontakt aufgenommen und Kontaktdaten wurden ausgetauscht. Nach wenige Emails wurde das erste Treffen vereinbart und die Abhandlung der Diplomarbeit mit Unterstützung von Intact war fixiert. Im gleichen Treffen wurde bereits das Thema der Diplomarbeit im groben besprochen.  
	
\subsection{Projektmanagement}
\label{sec:Projektmanagement}
Das Projekt wurde nach der Scrum--vorgehensweise durchgeführt. Allerdings wurde von der Scrum--vorgehensweise abgewichen, da manche Eigenschaften für unser Projekt keinen Sinn gemacht hätten, oder gar nicht funktioniert hätten.
	\subsubsection{Scrum}
	Anstatt ein Projekt am Anfang des Projektes komplett durchzuplanen und langfristige Meilensteine zu setzen, gibt es bei Scrum sogenannte Sprints. Ein Sprint ist ein Zeitintervall unter 4 Wochen, an welchen Beginn ein Ziel für diesen Sprint festgelegt wird, an diesem Ziel wird dann im Sprint gearbeitet. Nach jedem Sprint sollte ein Teil des Projekts fertig werden. Durch diese Herangehensweise, baut sich das fertige Projekt mit der Zeit von selbst auf. Wichtig bei Scrum sind Artefakte, Rollen und Meetings.
	\paragraph{Artefakte}
		Artefakte sind Dokumente oder Grafiken welche jeden Projektbeteiligten helfen Übersicht zu behalten. Die Wichtigsten Artefakte sind: Vision-Dokument, Product-Backlog, Product-Increment und der Sprint-Backlog.\\
		
			\textbf{Vision-Dokument}\\
			Das Visionsdokument befasst sich im groben worum es im Projekt geht. Es beschreibt den Zweck und das Ziel oder die Ziele des Projekts. Rahmenbedingungen wie zum Beispiel Budget oder Zeit werden ebenfalls im Visionsdokument festgehalten. Im Visionsdokument wird das geplante Produkt mit ähnlichen bereits existierenden Produkten anderer Unternehmen verglichen und es wird erwägt welchen Vorteil gegenüber den bereits existierenden Produkten existieren.
			Das Wichtigste am Visionsdokument ist, dass man sich von Anfang an das fertige Produkt vorstellen kann sodass keine Verwirrungen entstehen.
						
			\textbf{Product-Backlog}

			\textbf{Product-Increment}

			\textbf{Sprint-Backlog}
		
	\paragraph{Rollen}
		Bei Scrum wird das Team in Rollen eingeteilt, jede Rolle hat eine spezielle Funktionalität welche im Laufe des Projekts durchgeführt werden muss. Eingeteilt wird in Product Owner, ScrumMaster und das Team.
	\paragraph{Meetings}
		Meetings sind ein extrem wichtiger Teil des Scrumprozesses, solange sie gut geleitet werden und von jedem Teammitglied ernst genommen werden können sie die Effizienz enorm steigern. Essentielle Ereignisse sind das Sprint-planning-meeting, der Daily-Scrum, die Sprint-Retroperspective und der Sprint-Review. 
