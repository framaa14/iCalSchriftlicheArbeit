\documentclass[11pt]{scrartcl}

\usepackage{ucs}
\usepackage[utf8x]{inputenc}
\usepackage[T1]{fontenc}
\usepackage[ngerman]{babel}
\usepackage{graphicx}
\usepackage[automark]{scrpage2}

\newcommand{\titledate}[2][2.5in]{%
  \noindent%
  \begin{tabular}{@{}p{#1}@{}}
    \\ \hline \\[-.75\normalbaselineskip]
    #2
  \end{tabular} \hspace{1in}
  \begin{tabular}{@{}p{#1}@{}}
    \\ \hline \\[-.75\normalbaselineskip]
    Ort, am TT.MM.JJJJ
  \end{tabular}
}

\author{
  Wagner Dario
  \and
  Stering Marcel
  \and
  Franz Matthias
}

\title{Diplomarbeit}
\subtitle{iCal Web-Service}
\date{\today{}, Kaindorf a.d. Sulm}

\begin{document}
\begin{titlepage}
	\maketitle
\end{titlepage}

\section*{Eidesstattliche Erklärung}
\label{sec:eidesstattliche-erklaerung}
KRIEGEN WIR VON DER SCHULE

\vspace*{30px}

\titledate{Vor-/Zuname, Unterschrift}
\vspace*{30px}

\titledate{Vor-/Zuname, Unterschrift}
\vspace{30px}

\titledate{Vor-/Zuname, Unterschrift}


\section*{Abstract}
\label{sec:abstract}
Diese Diplomarbeit befasst sich mit einem Stück Software welche im Auftrag der Firma Intact GmbH angefertigt wurde. Das Ziel der Diplomarbeit ist es, AuditorInnen welche die bereits existierende Anwendung Ecert verwenden, Kalender immer und überall verfügbar zu machen. Erreicht wurde dies mit Verwendung des iCal-Formates welches von jeder Kalender-Applikation verwendet wird um Kalender anzuzeigen und zu speichern. Die Kalender der AuditorInnen werden gespeichert und nachdem man sich auf einer Webseite angemeldet hat, kann man auf alle seine Kalender zugreifen und in jegliche Kalender-Applikation einbinden. Somit müssen sich AuditorInnen nicht mehr darauf konzentrieren, dass alle ihre/seine Kalender auf dem Gerät sind, denn diese sind nun übers Internet erreichbar. 
\vspace{20px}
\linebreak
The subject of this thesis is a piece of software which was written on the behalf of Intact GmbH. The aim of this thesis is to offer auditors who already use Intact GmbHs own software, Ecert, the ability to access their calendars everywhere and anytime they want. This achievable because nearly every calendar-app uses the iCal-format to save the calendar. The iCal-format gets saved and the auditor just needs to login into a website and there they can find all their calendars ready to be integrated in their favorite calendar-app.

\section*{Vorwort}
\label{sec:vorwort}
Gründe für Themenwahl und persönlicher Bezug dazu.

\newpage
	\tableofcontents
\newpage

\section*{Danksagung}
\label{sec:danksagung}


\section{Einleitung}
\label{sec:einleitung}
\subsection{Kurzbeschreibung}
\label{sec:kurzbeschreibung}
\subsection{Vorgehensweise}
\label{sec:vorgehensweise}

\section{Projektmanagement und Organisation}
\label{sec:projektmanagement-und-organisation}
\subsection{Intact GmbH}
\label{sec:intact-gmbh}

\section{Technologien}
\subsection{Allgemeines}
Unsere verwendeten Technologien werden anschließend unter entsprechender beschrieben, wobei auf die wichtigsten, oder auch meist benutzten, genauer eingegangen wird, in Form einer Installation und erweiterten Beschreibung.
\subsection{Programmierung}
\begin{itemize}
\item Visual Studio 17 Community
\item .net Framework 4.6
\item asp.net
\item MSSQL
\item Entity Framework
\item iCal
\item Microsoft SQL Server management Studios
\item ReSharper
\item PostMan
\end{itemize}
\subsection{Kommunikation}
\begin{itemize}
\item Discord
\item Telegram
\end{itemize}
\subsection{Sharing}
\begin{itemize}
\item TFS
\item Discord
\item Google Drive
\end{itemize}
\subsection{Organisation}
\begin{itemize}
\item Trello
\item Discord
\end{itemize}
\subsection{Schriftliche Arbeit}
\begin{itemize}
\item LaTeX
\end{itemize}


\section{iCal-Format}
\label{sec:ical-format}
\subsection{Allgemeines}
\label{sec:ical-allgemeines}
\subsection{Einbindung in Kalender-Apps}
\label{sec:ical-einbindung-in-kalenderapps}
\subsection{Keywords}
\label{sec:ical-keywords}
Die wichtigsten und von uns verwendeten Keywords werden hier beschrieben und erkärt. Teilweise anhand von Beispielen falls diese sehr umfangreich sind. 


\section{Datenbank}
\label{sec:datenbank}
\subsection{MSSQL}
\label{sec:db-mssql}
\subsection{Aufbau}
\label{sec:db-aufbau}

\section{Parser}
\label{sec:parser}
\subsection{Aufgabe}
\label{sec:parser-aufgabe}
Die Aufgabe des Parsers ist es auf die Datenbank zuzugreifen und sich die, für das iCal Format notwendigen, Daten zu holen. Diese werden anschließend vom Parser in einen iCal String umgewandelt, damit der benutzte Kalender diesen verwerten kann und passende Termine erstellt. 

\subsection{Entity Framework}
\label{sec:parser-entity-framework}

\section{Webseite}
\label{sec:Webseite}
\subsection{Security}
\label{sec:Security}
Dieser Abschnitt behandelt das Security-Handling unserer Service-Webseite.
\subsubsection{Login Handling}
\label{sec:Login}
\subsubsection{Absicherung}
\label{sec:Absicherung}
\subsubsection{Two-Factor-Auth}
\label{sec:tfa}
\subsubsection{XSRF/CSRF Protection}
\label{sec:xss}
\subsubsection{Sql-Injection Protection}
\label{sec:sqli}
\subsubsection{Password Hashes}
\label{sec:hash}
\subsection{ASP.NET MVC}
\label{sec:MVC}
\subsubsection{Allgemeines MVC}
\label{sec:allgemein}

\subsubsection{Website Aufbau}
\label{sec:aufbau}
\subsubsection{Link generation}
\label{sec:link}
\subsubsection{Controller}
\label{sec:Controller}
\subsubsection{Views}
\label{sec:Views}
\subsubsection{Services}
\label{sec:Services}
\subsubsection{User Datenbank}
\label{sec:UserDB}
\begin{itemize}
	\item Salt
\end{itemize}
\label{sec:salt}

\section{Tests}
\label{sec:Tests}
\subsection{Allgemeines}
Hier werden alle von uns vorgenommenen Tests dokumentiert und beschrieben.

\subsection{iCal}
Software bezogene Tests

\subsection{Website-Security}
Penetration Test

\section{Literatur und Quellenverzeichnis}
\label{sec:literatur-quellenverzeichnis}

\section{Abbildungsverzeichnis}
\label{sec:abbildungsverzeichnis}

\section{Tabellenverzeichnis}
\label{sec:tabellenverzeichnis}

\section{Abkürzungsprotokoll}
\label{sec:abkuerzungsprotokoll}

\end{document}