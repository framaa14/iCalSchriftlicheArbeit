\documentclass[12pt]{scrartcl}

\usepackage{ucs}
\usepackage[utf8x]{inputenc}
\usepackage[T1]{fontenc}
\usepackage[ngerman]{babel}
\usepackage{graphicx}
\graphicspath{ {./images/} }
\usepackage[automark]{scrpage2}
\usepackage{color}
\usepackage{listings}
\usepackage{courier}
\usepackage{xcolor}
\usepackage{hyperref}
\usepackage{titlesec}

\setcounter{secnumdepth}{4}
\titleformat{\paragraph}
{\normalfont\normalsize\bfseries}{\theparagraph}{1em}{}
\titlespacing*{\paragraph}
{0pt}{3.25ex plus 1ex minus .2ex}{1.5ex plus .2ex}

% Nice Formatierung für C# Source Code [Anfang]
\definecolor{bluekeywords}{rgb}{0,0,1}
\definecolor{greencomments}{rgb}{0,0.5,0}
\definecolor{redstrings}{rgb}{0.64,0.08,0.08}
\definecolor{xmlcomments}{rgb}{0.5,0.5,0.5}
\definecolor{types}{rgb}{0.17,0.57,0.68}

\usepackage{listings}
\lstset{language=[Sharp]C,
captionpos=b,
%numbers=left, %Nummerierung
%numberstyle=\tiny, % kleine Zeilennummern
frame=lines, % Oberhalb und unterhalb des Listings ist eine Linie
showspaces=false,
showtabs=false,
breaklines=true,
showstringspaces=false,
breakatwhitespace=true,
escapeinside={(*@}{@*)},
commentstyle=\color{greencomments},
morekeywords={partial, var, value, get, set},
keywordstyle=\color{bluekeywords},
stringstyle=\color{redstrings},
basicstyle=\ttfamily\small,
}
% Nice Formatierung für C# Source Code [Ende]

%Titelseite
\newcommand{\titledate}[2][2.5in]{%
  \noindent%
  \begin{tabular}{@{}p{#1}@{}}
    \\ \hline \\[-.75\normalbaselineskip]
    #2
  \end{tabular} \hspace{1in}
  \begin{tabular}{@{}p{#1}@{}}
    \\ \hline \\[-.75\normalbaselineskip]
    Ort, am TT.MM.JJJJ
  \end{tabular}
}

\author{
  Wagner Dario
  \and
  Stering Marcel
  \and
  Franz Matthias
}

\title{Diplomarbeit}
\subtitle{iCal Web-Service}
\date{\today{}, Kaindorf a.d. Sulm}
%Titelseite

\begin{document}
\begin{titlepage}
	\maketitle
\end{titlepage}

\section*{Eidesstattliche Erklärung}
\label{sec:eidesstattliche-erklaerung}
KRIEGEN WIR VON DER SCHULE

\vspace*{30px}

\titledate{Vor-/Zuname, Unterschrift}
\vspace*{30px}

\titledate{Vor-/Zuname, Unterschrift}
\vspace{30px}

\titledate{Vor-/Zuname, Unterschrift}
\pagebreak

\section*{Abstract}
\label{sec:abstract}
Diese Diplomarbeit befasst sich mit einem Stück Software welche im Auftrag der Firma Intact GmbH angefertigt wurde. Das Ziel der Diplomarbeit ist es, AuditorInnen welche die bereits existierende Anwendung Ecert verwenden, Kalender immer und überall verfügbar zu machen. Erreicht wurde dies mit Verwendung des iCal-Formates welches von jeder Kalender-Applikation verwendet wird um Kalender anzuzeigen und zu speichern. Die Kalender der AuditorInnen werden gespeichert und nachdem man sich auf einer Webseite angemeldet hat, kann man auf alle seine Kalender zugreifen und in jegliche Kalender-Applikation einbinden. Somit müssen sich AuditorInnen nicht mehr darauf konzentrieren, dass alle ihre/seine Kalender auf dem Gerät sind, denn diese sind nun übers Internet erreichbar. 
\vspace{20px}
\linebreak
The subject of this thesis is a piece of software which was written on the behalf of Intact GmbH. The aim of this thesis is to offer auditors who already use Intact GmbHs own software, Ecert, the ability to access their calendars everywhere and anytime they want. This achievable because nearly every calendar-app uses the iCal-format to save the calendar. The iCal-format gets saved and the auditor just needs to login into a website and there they can find all their calendars ready to be integrated in their favorite calendar-app.
\pagebreak

\section*{Vorwort}
\label{sec:vorwort}
Gründe für Themenwahl und persönlicher Bezug dazu.

\newpage
	\tableofcontents
\newpage

\section*{Danksagung}
\label{sec:danksagung}


\section{Einleitung}
\label{sec:einleitung}
\subsection{Kurzbeschreibung}
\label{sec:kurzbeschreibung}
\subsection{Vorgehensweise}
\label{sec:vorgehensweise}

\section{Projektmanagement und Organisation}
\label{sec:projektmanagement-und-organisation}
\subsection{Intact GmbH}
\label{sec:intact-gmbh}

\section{Technologien}
\subsection{Allgemeines}
Unsere verwendeten Technologien werden anschließend unter entsprechender beschrieben, wobei auf die wichtigsten, oder auch meist benutzten, genauer eingegangen wird, in Form einer Installation und erweiterten Beschreibung.
\subsection{Programmierung}
\subsubsection {Visual Studio 17 Community}
Visual Studio ist eine Entwicklungsumgebung, für verschiedenste Programmiersprachen, der Firma Microsoft. Die Version 15 / 2017 ist die aktuellste Version und bietet neue Funktionen und Verbesserungen. Unter anderem die voll umfängliche Unterstützung der ASP.NET Core und .NET Core Entwicklung. Die aktuelle Version unterstützt folgende Sprachen:
\begin{itemize}
\item Visual Basic .NET
\item C
\item C++
\item C\#
\item F\#
\item Typescript
\item Python
\item HTML
\item JavaScript
\item CSS
\end{itemize}

Da der Hauptteil unserer Diplomarbeit in der Objekt Orientierten Programmiersprache C\# geschrieben wurde, hat das Entwicklungsteam Visual Studio 2017 Community verwendet. Hierbei war es uns wichtig, dass jeder von uns die selbe "Jahres-Version", in diesem Fall 2017, verwendet, da es zwischen den Versionen kleine Unterscheide, welche zu einem Problem führen könnten, gibt. Ein gravierender Unterschied wäre die Syntax eines Propertys zwischen Version 2013 und 2017. 

\begin{lstlisting}[caption=Syntax Unterschied: Property , label=lst:test]
// Visual Studio 2013 Code
  private string m_Beispiel;
  public string Beispiel
  {
      get { return m_Beispiel; }
      set { m_Beispiel = value; }
  }

// Visual Studio 2017 Code
  private string m_Beispiel;
  public string Beispiel
  {
      get => m_Beispiel;
      set => m_Beispiel = value;
  }
\end{lstlisting}

\subsubsection {.net Framework 4.6}
Am Anfang der Diplomarbeit wurde mit der Firma im Laufe eines Meetings festgelegt, dass bei der Entwicklung des Webservices .net Framework 4.6 verwendet werden soll um die Kompatibilät mit ihren .net Projekten zu garantieren.

\subsubsection {asp.net}
Da das Ziel der Diplomarbeit ein Webservice unter C\# ist, wurde ASP.NET verwendet. ASP.NET ist Teil des .net Framework, mit ihm lassen sich Webservices oder auch Webanwendungen einfach entwickeln. ASP.NET kommt bei 11.8\% aller aktiven Webseiten zum Einsatz und befindet sich deshalb auf dem 2ten Platz nach der Programmiersprache PHP. 
\url {https://w3techs.com/technologies/overview/programming_language/all} 
% DAS DER LINK SO DRIN IS IS NUR VORÜBERGEHEND DAMIT I MA DIE SEITE MERK
\subsubsection {MSSQL}
MSSQL ist KEIN Teil der finalen Diplomarbeit und wurde nur zu Testzwecken verwendet. Im Laufe der Entwicklung wurde von Teammitglied Matthias Franz und Marcel Stering ein Raspberry PI als Datenbank aufgesetzt um einige Tests durchzuführen. Dies wurde mit Microsoft SQL Server verwirklicht. 
\subsubsection {Microsoft SQL Server management Studios}
Bei der Microsoft SQL Server entwicklung kam Microsoft SQL Server management Studios zum Einsatz, die Aufgabe des Management Studios war es den Server zu konfigurieren und zu verwalten. 
\subsubsection {Entity Framework}
Das Entity Framework ist ein Großteil des Projektparts "Parser" gewesen. Das Entity Framework wird angewandt um den Zugriff auf die Datenbank zu erleichtern. Es dient zur objektrationalen Abbildung auf .NET Objektstrukturen. Auf die Funktionsweise des EFs wird im Parser genauer eingegangen.

\subsubsection {iCal}
iCal ist das Format in dem ein Kalender gespeichert wird. Das Format wird unter einer eigenen Überschrift im Laufe der schriftlichen Arbeit genauer erklärt. 
\subsubsection {ReSharper}
\subsubsection {PostMan}


\subsection{Kommunikation}
\subsubsection {Discord}
Um im Laufe des praktischen Teils der Diplomarbeit die Übersicht zu behalten und alles zu organisieren wurde Discord verwendet. Discord hat viele Funktionen welche die Kommunikation im Team erleichtern. Discord bietet dem Benutzer an einen oder mehrere gratis Server zu erstellen. Ein Server kann aus Text und Sprachchannels bestehen. In einem Textchannel können festgelegte Personen schreiben und in einem Sprachchannel über Mikrofon miteinander reden. Falls wir also Teamintern etwas zu besprechen hatten oder falls Probleme auftraten die wir selbst lösen konnten bat uns Discord die perfekte Kommunikationsfläche. 

Da wir als Gruppe mehrere Projekte haben haben wir einen "Projektserver". In diesem Projektserver haben wir einen Text und Sprach Channel für die Diplomarbeit. Im Text Channel werden kleine Probleme, die schnell geklärt werden können, besprochen und Files ausgetauscht. Im Sprach Channel werden gröbere Probleme besprochen oder wenn nötig Planänderungen. 

\subsubsection {Telegram}
Telegram wurde nicht regelmäßig verwendet, es war eher eine Backup Chat-Application. 

\subsection{File Sharing}
\subsubsection {TFS}
Der Microsoft Team Foundation Server ist unsere Code-Sharing Technologie. Da unser Auftraggeber, die Firma Intact GmbH oder Intact Systems, mit dieser Technologie arbeitet haben wir bei einem der ersten Treffer TFS für Code Sharing gewählt. Wir hatten einige Probleme mit dem TFS wodurch oft einzelne Teile des Projekts entwickelt wurden und dann in ein Projekt zusammengeführt wurden. Die Probleme waren unter anderem, dass die Firma eine Zeit lang gebraucht hat um den Server zur Verfügung zustellen aber auch, dass das Verbinden mit dem Server manchmal nicht geklappt hat. 
\subsubsection {Discord}
Wie bereits bei den Technologien erwähnt haben wir auf einem Discord Server einen Text Channel eingerichtet. Dieser eignet sich nicht nur um miteinander zu schreiben sondern kann auch dafür genutzt werden mit anderen Benutzer Files zu teilen. 
\subsubsection {Google Drive}
Google Drive ist ein von Google bereitgestellter Cloud Service um Dokumente freizugeben und Online zu bearbeiten.

Mithilfe von Google Drive wurde an Präsentationen und Projekten gearbeitet. Durch Google Docs und Google Präsentation fällt es leicht mit mehreren Personen gleichzeitig an einem Dokument zu arbeiten. Durch Google Drive wurden von uns Dokumente wie die IVM Matrix, den Projektstrukturplan, die Meetings und die SCRUM Sprints erstellt und an alle Mitglieder geteilt. 
\subsection{Organisation}
\subsubsection {Trello}
\subsubsection {Discord}


\subsection{Schriftliche Arbeit}
\subsubsection {LaTeX}


\section{iCal-Format}
\label{sec:ical-format}
\subsection{Allgemeines}
\label{sec:ical-allgemeines}
\subsection{Einbindung in Kalender-Apps}
\label{sec:ical-einbindung-in-kalenderapps}
\subsection{Keywords}
\label{sec:ical-keywords}
Die wichtigsten und von uns verwendeten Keywords werden hier beschrieben und erkärt. Teilweise anhand von Beispielen falls diese sehr umfangreich sind. 


\section{Datenbank}
\label{sec:datenbank}
\subsection{MSSQL}
\label{sec:db-mssql}
\subsection{Aufbau}
\label{sec:db-aufbau}

\section{Parser}
\label{sec:parser}
\subsection{Aufgabe}
\label{sec:parser-aufgabe}
Die Aufgabe des Parsers ist es auf die Datenbank zuzugreifen und sich die, für das iCal Format notwendigen, Daten zu holen. Diese werden anschließend vom Parser in einen iCal String umgewandelt, damit der benutzte Kalender diesen verwerten kann und passende Termine erstellt. 

\subsection{Entity Framework}
\label{sec:parser-entity-framework}
\subsubsection {Funktionsweise}
Mithilfe des Entity Framework lässt sich eine Datenbankstruktur innerhalb des Projekts mit Klassen darstellen. Wenn auf eine dieser Klassen in Form einer Value-Abfrage zugegriffen oder durch sonstige GET/SET Methoden, wird durch das Entity Framework ein Datenbank Zugriff durchgeführt. 
Um die Funktionsweise genauer zu verstehen folgt ein Beispiel mit einer Datenbank in welcher Autos gespeichert werden:
HIER KOMMT DANN EIN BEISPIEL
\subsubsection {Anwendung}
HIER KOMMT DANN DIE INSTALLATION

\section{Webseite}
\label{sec:Webseite}
\subsection{Security}
\label{sec:Security}
In diesem Abschnitt beschäftigen wir uns mit der Security der Website.
Wir behandeln wie man das sichere einloggen in eine Website gewähren kann, 
wie man sich vor XSS/CSRF schützen kann, wie man verhindert das eine SQL Injection
möglich ist und wie man Passwörter speichert. Dazu werden wir einige Code Beispiele anführen.

\subsubsection{Login Handling}
\label{sec:Login}
\subsubsection{Absicherung}
\label{sec:Absicherung}
\subsubsection{Two-Factor-Auth}
\label{sec:tfa}
\subsubsection{XSRF/CSRF Protection}
\label{sec:xss}
\subsubsection{Sql-Injection Protection}
\label{sec:sqli}
\subsubsection{Password Hashes}
\label{sec:hash}
\subsection{ASP.NET MVC}
In diesem Abschnitt beschäftigen wir uns mit ASP.NET MVC mit der unsere Website aufgebaut ist. Wir besprechen die Grundindention von MVC und was MVC ist. Wie die Website aufgebaut wurde werden wir anhand Code auszügen zeigen. Die beim MVC bekannten Views Controllers und Services werden aufgezeigt und erklärt. Ebenfalls wird behandelt wie die Links zu den Kalendern erzeugt und zur Verfügung gestellt werden. 
\label{sec:MVC}
\subsubsection{Allgemeines MVC}
\label{sec:allgemein}

\subsubsection{Website Aufbau}
\label{sec:aufbau}
\subsubsection{Link generation}
\label{sec:link}
\subsubsection{Controller}
\label{sec:Controller}
\subsubsection{Views}
\label{sec:Views}
\subsubsection{Services}
\label{sec:Services}
\subsubsection{User Datenbank}
\label{sec:UserDB}
\begin{itemize}
	\item Salt
\end{itemize}
\label{sec:salt}

\section{Tests}
\label{sec:Tests}
\subsection{Allgemeines}
Hier werden alle von uns vorgenommenen Tests dokumentiert und beschrieben.

\subsection{iCal}
Software bezogene Tests

\subsection{Website-Security}
Penetration Test

\section{Literatur und Quellenverzeichnis}
\label{sec:literatur-quellenverzeichnis}

\section{Abbildungsverzeichnis}
\label{sec:abbildungsverzeichnis}

\section{Tabellenverzeichnis}
\label{sec:tabellenverzeichnis}

\section{Abkürzungsprotokoll}
\label{sec:abkuerzungsprotokoll}

\end{document}
