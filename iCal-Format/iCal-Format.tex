%TODO
%iCal Ding als Code-Listing?

\renewcommand{\theauthor}{Matthias Franz}
\section{iCal}
\label{sec:iCal}
Dieses Kapitel befasst sich mit dem iCal-Dateiformat welches einen großen Teil in dieser Diplomarbeit einnimmt. Es wird behandelt wie eine iCal-Datei aufgebaut ist und weshalb iCal in diesem Projekt verwendet wurde.

\subsection{Was ist iCal?}
\label{sec:wasIstiCal?}
iCal ist ein Dateiformat, welches dazu verwendet wird um Kalender zu speichern. Fast jede Kalenderanwendung verwendet zur Speicherung und Manipulation ihrer Kalender iCal. Als Datei hat eine iCal-Datei die Endung .ics. Eine .ics Datei ist von Menschen lesbar und leicht veränderbar, was die Arbeit mit iCal-Dateien um einiges vereinfacht. \\
iCal ist ein MIME-Typ, dies ermöglicht es iCal-Dateien über jegliche Methoden zu versenden.\\ \textcite{iCalDocumentation} 

\subsection{Warum wurde iCal verwendet?}
\label{sec:warumWurdeiCalVerwendet?}
iCal wurde verwendet, da der Großteil der Kalenderprogramme das iCal-Format verwenden und es viele Ressourcen rund um iCal gibt, was den Umgang damit deutlich vereinfacht. Weiters sind die Grundlagen einer iCal-Datei schnell verstanden wegen des einfachen Aufbau einer Datei.

\subsection{Aufbau einer iCal-Datei}
\label{sec:aufbauEineriCalDatei}
iCal-Dateien sind in einer Key-Value-Struktur aufgebaut, wobei sich jedes Key-Value-Paar in einer eigenen Zeile befindet. Eine iCal-Datei kann aus mehreren Kalendern bestehen und ein Kalender kann mehreren Objekten bestehen, die wichtigsten sind: Event-, To-do- und Journal-Elemente, welche genauer im Kapitel \ref{sec:keywords}.\\\\
\texttt{BEGIN:VCALENDAR\\
 VERSION:2.0\\
 PRODID:-//hacksw/handcal//NONSGML v1.0//EN\\
 BEGIN:VEVENT\\
 UID:19970610T172345Z-AF23B2@example.com\\
 DTSTAMP:19970610T172345Z\\
 DTSTART:19970714T170000Z\\
 DTEND:19970715T040000Z\\
 SUMMARY:Bastille Day Party\\
 END:VEVENT\\
 END:VCALENDAR}\\\\
 
Wie man sieht ist eine .ics Datei hierarchisch aufgebaut. Eine Datei muss mit BEGIN:VCALENDAR beginnen, wenn ein Kalender mit BEGIN:VCALENDAR begonnen wird muss er wie jedes andere Element einer iCal-Datei auch wieder geschlossen werden um die beinhalteten Elemente einordnen zu können. Die VCALENDAR Eigenschaft kann viele Attribute haben welche das Verhalten des Kalenders verändern, die meisten dieser Attribute sind allerdings nicht für diese Diplomarbeit relevant und wurden deshalb weggelassen. In einem VCALENDAR Element kann man dann entweder ein Event-, To-do oder Journal-Element erstellen, es gibt noch weiter erstellbare Elemente, diese sind aber für diese Diplomarbeit nicht von Relevanz. Wie im Beispiel angeführt wird im VCALENDAR ein VEVENT erstellt. Dieses VEVENT muss dann wie der VCALENDAR und alle anderen Attribute wieder geschlossen werden. Im Beispiel sieht man, dass Events auch mehrere Attribute hat welches das Verhalten des Events ändern, zum Beispiel SUMMARY, beschreibt was in der Kalenderapplikation in diesem Termin stehen würde.\\

\pagebreak
\renewcommand{\theauthor}{Dario Wagner}
\subsection{Keywords}
\label{sec:keywords}
des ist nur da, damit ich es referenzieren kannst, wennst die überschrift ändern willst änder des in der Referenz von "Aufbau einer iCal-Datei auch mit.