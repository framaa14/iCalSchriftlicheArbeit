%%%% Time-stamp: <2013-02-25 10:31:01 vk>


\chapter*{Abstract - DE}
\label{cha:abstract}

Diese Diplomarbeit befasst sich mit einem Stück Software, welche im Auftrag der Firma Intact GmbH angefertigt wurde. Das Ziel der Diplomarbeit ist es, AuditorInnen, welche die bereits existierende Anwendung Ecert verwenden, Kalender immer und überall verfügbar zu machen. Erreicht wurde dies durch Verwendung des iCal-Formates, welches von jeder Kalender-Applikation verwendet wird um Kalender anzuzeigen und zu speichern. Die Kalender der AuditorInnen werden gespeichert und nachdem man sich auf einer Webseite angemeldet hat, kann man auf alle seine Kalender zugreifen und in jegliche Kalender-Applikation einbinden. Somit müssen sich AuditorInnen nicht mehr darauf konzentrieren, dass alle ihre/seine Kalender auf dem Gerät sind, denn diese sind nun übers Internet erreichbar. 
\vspace{20px}
\linebreak
\pagebreak

\pagebreak

\chapter*{Abstract - EN}
\label{cha:abstract}
The subject of this thesis is a piece of software which was written on the behalf of Intact GmbH. The aim of this thesis is to offer auditors who already use Intact GmbHs own software, Ecert, the ability to access their calendars everywhere and anytime they want. This is achievable because nearly every calendar-app uses the iCal-format to store the calendar. The iCal-format gets stored and the auditor just needs to login into a website and there they can find all their calendars ready to be integrated in their favorite calendar-app.

%\glsresetall %% all glossary entries should be used in long form (again)
%% vim:foldmethod=expr
%% vim:fde=getline(v\:lnum)=~'^%%%%\ .\\+'?'>1'\:'='
%%% Local Variables:
%%% mode: latex
%%% mode: auto-fill
%%% mode: flyspell
%%% eval: (ispell-change-dictionary "en_US")
%%% TeX-master: "main"
%%% End:
